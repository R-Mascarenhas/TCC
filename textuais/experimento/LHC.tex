\chapter{Física de Altas Energias e o CERN}\label{cap:LHC}
	
Desde os anos setenta, os físicos de partículas têm descrito a estrutura fundamental da matéria usando uma elegante série de equações denominado Modelo Padrão. O modelo tenta descrever tudo que pode ser observado no universo, a partir de alguns blocos básicos chamados partículas fundamentais \cite{cernphisics}.

Portanto, para a comprovação experimental dessa teoria, equipamentos que são capazes de colidir partículas em altas energias foram construídos, recriando um ambiente onde é possível observar, mais profundamente, as partículas fundamentais e seus processos de interação. Desde 1911, quando o primeiro acelerador de partículas foi criado pelo físico britânico Ernest Rutherford \cite{aceleradores2015}, aceleradores vem sendo construídos com energias de colisão cada vez maiores. Em 2008, o \ac{CERN} inaugurou o maior acelerador de partículas do mundo, o \ac{LHC}, projetado para colidir prótons a uma energia de centro de massa de 14 TeV.
	
Esta seção é dedicada a ambientar e dar uma visão geral sobre qual é o objeto de estudo e onde este estudo é realizado, ou seja, será feita uma breve explicação sobre o Modelo Padrão, o CERN, o experimento LHC e o detector ATLAS.


\section{Modelo Padrão}

O Modelo Padrão das partículas elementares, não é exatamente um modelo, mas sim uma teoria muito bem fundamentada, considerada a melhor teoria sobre a natureza da matéria por muitos físicos \cite{moreira2009modelo}.

De acordo com o Modelo Padrão, todas  as  partículas  podem  ser  classificadas  em  Bósons  e  Férmions, sendo que os primeiros não obedecem o Princípio de Exclusão de Pauli, que é um princípio da mecânica quântica, que afirma que dois férmions idênticos não podem ocupar o mesmo estado quântico simultaneamente. Este modelo também descreve os mecanismos de interação regidos pelas forças: eletromagnética, fraca e forte; a única força não abrangida por esta teoria é a força gravitacional. \cite{perkins2000introduction}

As partículas constituintes da matéria podem ser divididas e nomeadas como segue:
\begin{itemize}
  \item Férmions: partículas que constituem a matéria e são subdivididas em léptons e quarks.
  \item Léptons: elétron, múon, tau e seus neutrinos e suas antipartículas.
  \item Quarks são: up, down, charm, strange, top e bottom e suas antipartículas.
\end{itemize}

%\begin{figure}[!h]
%  \centering
%  \includegraphics[width=10cm]{./textuais/experimento/figuras/fig3.png}\\
%  \caption{Modelo Padrão. Extraído de \cite{fehling2011standard}.}
%  \label{fig:2T01}
%\end{figure}

As partículas transportadoras de força que mediam as interações entre partículas e são: glúon(força forte), fóton(força eletromagnética), bósons W e Z(força fraca) e o bóson de Higgs(responsável pela existência de massa inercial).

Um resumo das informações das partículas do Modelo Padrão é apresentado na Figura~\ref{fig:2T02}.

\begin{figure}[!h]
  \centering
  \includegraphics[width=12cm]{./textuais/experimento/figuras/fig3.png}\\
  \caption{Modelo Padrão. Extraído de \cite{modelopadrao}.}
  \label{fig:2T02}
\end{figure}

\section{CERN}

No CERN, físicos e engenheiros trabalham em conjunto com o objetivo de investigar a estrutura fundamental do universo. Fundada em 1954, o laboratório CERN foi construído na fronteira franco-suíça, em Genebra. Ele foi um dos primeiros empreendimentos conjuntos da Europa e tem agora 21 Estados membros \cite{cernwebabout}.

O principal foco desta organização é a física de partículas que abrange estudos como: composição, raios cósmicos, matéria escura, dimensões extras, grávitons, minúsculos buracos negros, íons pesados, plasma quark-glúon, entre outros \cite{cernphisics}.

A necessidade de comprovar as teorias e estudar as partículas de maneira mais profunda tornou a construção do acelerador de partículas \ac{LHC} imprescindível. Nele, feixes de prótons são acelerados em direções opostas até atingirem altas energias e colidirem uns com os outros. A Seção~\ref{sec:LHC} abordará melhor a estrutura deste aparato.

Com o intuito de 'ler' e armazenar as informações geradas nas colisões dentro do acelerador, faz-se necessária a utilização de detectores, no local exato das colisões entre os feixes. Atualmente. o LHC conta com alguns detectores como: \ac{ALICE}, \ac{ATLAS}, \ac{CMS}, \ac{LHCb}. A Figura~\ref{fig:2T04} mostra a localização dos detectores (ALICE, ATLAS, CMS e LHCb) no LHC.

\begin{figure}[!h]
  \centering
  \includegraphics[width=12cm]{./textuais/experimento/figuras/fig1.png}\\
  \caption{Uma visão geral do experimento LHC. Extraído de CERN \cite{cernwebabout}.}
  \label{fig:2T04}
\end{figure}

Na Seção~\ref{sec:ATLAS}, o detector de ATLAS será melhor detalhado, uma vez que os dados utilizados nessa dissertação foram gerados por esse detector.

\section{LHC}\label{sec:LHC}

O LHC,, Figura~\ref{fig:2T03}, tem cerca de 27 km de circunferência. Ele acelera prótons ou íons que viajam em direções opostas, e são colocados para colidir \cite{lefevre2009lhc}.

Um acelerador só pode acelerar certos tipos de partícula: em primeiro lugar, esses elementos precisam ter carga, uma vez que os feixes são manipulados por dispositivos eletromagnéticos que só podem influenciar as partículas carregadas; e, em segundo lugar, exceto em casos especiais, estas não podem decair. Isso limita o número de partículas que podem ser acelerados para elétrons, pósitrons, prótons e íons. É necessário acrescentar que em um acelerador circular, como o LHC, partículas pesadas, como prótons, têm uma perda de energia, através de radiação síncrotron, muito menor que partículas leves, como elétrons. Portanto, para obter colisões com energias muito elevadas, o LHC faz uso de prótons.

%
\begin{figure}[!h]
  \centering
  \includegraphics[width=12cm]{./textuais/experimento/figuras/fig2.png}\\
  \caption{O LHC é o maior e mais poderoso acelerador de partículas do mundo. Extraído de \cite{cernwebLHC}.}
  \label{fig:2T03}
\end{figure}
%

O complexo de aceleradores do CERN é uma sucessão de mecanismos que aceleram partículas a energias cada vez maiores. Cada um desses instrumentos aumenta a energia de um feixe de partículas, antes de injetar o feixe no LHC, propriamente dito.

Com a ajuda de um campo eletromagnético, os prótons dos átomos de hidrogênio são separados dos elétrons. Estes prótons, primeiramente, são acelerados a energia de 50 MeV pelo \ac{LINAC 2}. Esse feixe é então injetado no \ac{PSB}, que o leva a energia de 1,4 GeV, seguido pelo \ac{PS}, que o impulsiona a 25 GeV. Esses prótons são enviados para o \ac{SPS}, onde eles são acelerados para 450 GeV. No LHC, o último elemento nesta cadeia, feixes de partículas são aceleradas até a energia recorde de 7 TeV por feixe, nominal de operação, com que colidem \cite{cernwebAccelerator}.

%
No ano de 2015, depois de passar por manutenção e modificações, o LHC retomou as atividades preparado para atingir energia de colisão no centro de massa de 13 TeV, aproximadamente o dobro da energia que vinha trabalhando. Este \emph{upgrade} tem o intuito de alcançar novos resultados e descobertas, uma vez que funcionará a uma energia nunca antes alcançada.

\section{ATLAS}\label{sec:ATLAS}

Com 46 metros de comprimento, 25 metros de altura e 25 metros de largura, e 7 mil toneladas, o detector ATLAS, mostrado na Figura~\ref{fig:2T05}, é o maior detector de partículas já construído. Ele se situa em uma caverna a 100 metros de profundidade perto do prédio principal do CERN, como pode ser visto na Figura~\ref{fig:2T06}, próximo da cidade de Meyrin, na Suíça \cite{cernwebAtlas}.

\begin{figure}[!h]
	\centering
	\includegraphics[width=12cm]{./textuais/atlas/figuras/fig1.png}\\
	\caption{Modelo computacional do Detector ATLAS. Extraído de (www.atlas.ch).}
	\label{fig:2T05}
\end{figure}

\begin{figure}[!h]
	\centering
	\includegraphics[width=10cm]{./textuais/atlas/figuras/fig2.png}\\
	\caption{Vista subterrânea do detector ATLAS . Extraído de (www.atlas.ch).}
	\label{fig:2T06}
\end{figure}

No centro do detector ATLAS, feixes de partículas do LHC colidem gerando produtos da colisão sob a forma de novas partículas, que se espalham em todas as direções. Como este é um detector de uso geral, precisa ser capaz de identificar os mais diversos tipos de partículas. O detector contém seis subsistemas de detecção diferentes dispostos em camadas ao redor do ponto de colisão, no intuito de gravar as trajetórias, momentos e energias das partículas, permitindo, assim, que sua identificação individual seja possível.

Esta seção tem o intuito de prover uma visão geral deste detector, suas característica básicas, detalhes internos e funcionamento, visando um melhor entendimento do ambiente onde está inserida essa dissertação.

\subsection{Sistemas de Coordenadas}\label{subsec:sist_coord}

O detector \ac{ATLAS} tem  formato cilíndrico e, para a identificação da posição das partículas no detector, utiliza-se um sistema de coordenadas pre-estabelecido. Este sistema define o ponto de interação nominal como a origem do sistema de coordenadas; o eixo \emph{z} é definido pela direção do feixe e o plano \emph{xy} é transversal à direção do feixe. Alternativamente, usando coordenadas cilíndricas, o ângulo $\phi$ é medido, como de costume, em torno do eixo do feixe, e o ângulo polar $\theta$ é o ângulo a partir do eixo do feixe \cite{aad2008atlas}. Por fim, a \emph{pseudorapidez} $\eta  =  - \ln \tan \left( {\frac{\theta }{2}} \right)$, sendo $\eta$ e $\phi$ as principais coordenadas do ATLAS, como mostrado na Figura~\ref{fig:2T07}.

\begin{figure}[!h]
	\centering
	\includegraphics[width=12cm]{./textuais/atlas/figuras/sistema_de_coordenadas.pdf}\\
	\caption{Sistema de coordenadas do Detector ATLAS. Extraído de \cite{dos2006sistema}.}
	\label{fig:2T07}
\end{figure}

\subsection{Perfil dos Eventos do ATLAS}\label{subsec:perfil_eve}

O perfil dos eventos de interesse do ATLAS e suas particularidades ditaram as características de construção e modulação do detector, uma vez que a identificação dessas partículas é feita pelas características de sua assinatura, que é uma marca particular, deixada no aparato. Cada componente deste equipamento foi especificado para detectar um conjunto de propriedades das partículas \cite{Lippmann:2011bb}.

Com o conhecimento sobre as peculiaridades de cada parte do detector, podemos entender como é o perfil de alguns desses eventos de interesse, mostrados na Figura~\ref{fig:2T15} e traduzidos abaixo:

\begin{itemize}
  \item Detector de traço: múons, prótons, elétrons, píons e káons;
  \item Calorímetro eletromagnético: múons, elétrons, fótons, prótons, píons e káons;
  \item Calorímetro hadrônico: múons, prótons, nêutrons, píons e káons.
  \item Câmara de múons: Múons;
\end{itemize}

\begin{figure}[h!]
	\centering
	\includegraphics[width=12cm]{./textuais/atlas/figuras/assinatura_particulas_atlas.png}\\
	\caption{Modelo computacional da assinatura das partículas no detector ATLAS. Extraído de (cds.cern.ch).}
	\label{fig:2T15}
\end{figure}

\subsection{Detector Interno}

O \ac{ID} do ATLAS é composto de três partes: uma seção cilíndrica, chamada de Barril(\emph{Barrel}), que cobre a região central $\left( {\left| \eta  \right| \le 1} \right)$, e duas regiões em forma de disco, chamadas de tampas(\emph{Endcaps}), abrangendo as regiões $\left( {1 \le \left| \eta  \right| < 2.5} \right)$.

Os subprodutos das colisões primeiramente cruzam o tubo de feixe, em seguida, as três camadas do Detector de Pixels  (\ac{SPD}), quatro camadas do Detector de Traços baseado em semicondutores (\ac{SCT} e 36 tubos do Detector de Radiação de Transição (\ac{TRT} \cite{barberis2000atlas}.

\begin{itemize}
\item Detector de Pixels  (\ac{SPD}): Esse detector fornece medidas em duas dimensões com alta precisão perto do ponto de interação, que são especialmente importantes para a caracterização de partículas de decaimentos semi-leptônicos;

\item Detector de Traços baseado em semicondutores (\ac{SCT}): Faz 4 pares de medidas por traço e, combinado com o SPD, provê medidas de momento, parâmetro de impacto e posição de vértice;

\item Detector de Radiação de Transição (\ac{TRT}): Esse detector contribui de maneira significativa para a medida precisa do momento de todos os traços, bem como, proporciona uma capacidade inerente de identificação de elétron. \cite{benekos2003atlas}.
\end{itemize}

A Figura~\ref{fig:2T08} mostra a disposição dos detectores e a Figura~\ref{fig:2T09} mostra um corte transversal do ID.

\begin{figure}[h!]
	\centering
	\includegraphics[width=12cm]{./textuais/atlas/figuras/ID1.png}\\
	\caption{Modelo computacional do ID. Extraído de (cds.cern.ch).}
	\label{fig:2T08}
\end{figure}

\begin{figure}[h!]
	\centering
	\includegraphics[width=12cm]{./textuais/atlas/figuras/ID2.png}\\
	\caption{Modelo computacional do ID - corte transversal. Extraído de (cds.cern.ch).}
	\label{fig:2T09}
\end{figure}

\subsection{Calorímetros}

O calorímetro é um dispositivo que absorve toda a energia cinética de uma partícula, que ao colidir com seu material inicia um chuveiro de partículas, cuja interação fornece, ao fim da cadeia, um sinal eletrônico proporcional ao valor da energia depositada \cite{das1994introduction}.

Na interação com o calorímetro cria-se um processo em cascata, onde partículas secundárias são produzidas ao longo do detector. Uma fração dessa energia é entregue na forma de luz de cintilação que produz um sinal detectável.

As características básicas dos calorímetros são:
\begin{itemize}
\item Calorímetros podem ser sensíveis tanto a partículas neutras quanto a carregadas;
\item Pode ser utilizado para identificação de partículas, uma vez que há diferenças na forma de deposição de energia para elétrons, múons e hádrons, por exemplo;
\item Permite tanto medida da energia quanto de trajetória das partículas, devido à sua segmentação;
\item Tempo de resposta rápido (menor que 50 ns), adequando-se a um ambiente com alta taxa de eventos \cite{peralva2012detecccao}.
\end{itemize}

\subsubsection{Chuveiros Eletromagnéticos e Hadrônicos}\label{subsec:chuveiros}

Em física de altas energias podemos destacar dois tipos de chuveiros (ou cascatas) \cite{grupen2008particle}:

\begin{itemize}
\item Chuveiros eletromagnéticos (Figura~\ref{fig:2T10}): são iniciados por elétrons ou fótons com alta energia ao passarem pelo calorímetro. Essas partículas carregadas sofrem interações criando fótons que, por sua vez, se convertem em pares elétron-pósitron. Essa cascata aumenta até a energia dos elétrons ser menor que uma energia crítica;
\item Chuveiros hadrônicos (Figura~\ref{fig:2T11}): decorrem do comprimento de interação nuclear, e geralmente são muito maiores que os chuveiros eletromagnéticos. Seu desenvolvimento lateral é causado pela grande transferência de momento típica de interações nucleares; e são basicamente compostos por píons.
\end{itemize}

\begin{figure}[h!]
	\centering
	\includegraphics[width=13cm]{./textuais/atlas/figuras/photon.png}\\
	\caption{Simulação computacional utilizando algoritmo Corsika do Chuveiro Eletromagnético (100GeV), (a) vista lateral e (b) vista frontal.}
	\label{fig:2T10}
\end{figure}

\begin{figure}[h!]
	\centering
	\includegraphics[width=13cm]{./textuais/atlas/figuras/proton.png}\\
	\caption{Simulação computacional utilizando algoritmo Corsika do Chuveiro Hadrônico (100GeV), (a) vista lateral e (b) vista frontal.}
	\label{fig:2T11}
\end{figure}

\subsubsection{Calorímetro Eletromagnético}\label{subsec:cal_ele}

O \ac{EM} \cite{calorimeter2008construction} é composto de absorvedores de chumbo e eletrodos intercalados em forma de acordeão, sendo utilizado Argônio líquido como material ativo. Esse aparato compõe a parte mais interna no sistema de calorimetria do ATLAS.

\begin{figure}[h!]
	\centering
	\includegraphics[width=12cm]{./textuais/atlas/figuras/liquidargon_segmentation.png}\\
	\caption{Modelo computacional do Calorímetro Eletromagnético. Extraído de \cite{francavilla2012atlas}.}
	\label{fig:2T12}
\end{figure}

A construção desse calorímetro foi dividida em três camadas, sendo a primeira mais segmentada, no intuito de efetuar uma localização precisa da partícula; a segunda é mais profunda e menos segmentada que a primeira; e, por fim, a terceira camada é a menos segmentada e tem a função de absorver completamente a energia da partícula incidente. Essas divisões e segmentações podem ser vistas na Figura~\ref{fig:2T12}.

Como existem perdas de informação devido ao 'material morto' (fios, encapamentos, etc) o calorímetro EM possui um pré-irradiador, que atua na recuperação dessas informações.

\subsubsection{Calorímetro Hadrônico}\label{subsec:cal_had}

O \ac{HAD} do detector ATLAS, chamado de Calorímetro de Telhas (do inglês \emph{Tile Calorimeter}, ou \emph{TileCal}), utiliza placas cintiladoras, em formato de telha, como material ativo e, como material absorvedor, faz o uso de placas de aço com baixo carbono. Esse equipamento é subdivido em 3 partes, como pode ser visto na Figura~\ref{fig:2T13}: o barril \emph{(Tile Barrel)}, situado no região central e dois barris estendidos \emph{(Tile Extended Barrel)}, um em cada lateral do barril. O Tilecal também possui 3 camadas, cada uma segmentada de uma forma diferente \cite{aad2010readiness}.

\begin{figure}[h!]
	\centering
	\includegraphics[width=12cm]{./textuais/atlas/figuras/TileCal.png}\\
	\caption{Modelo computacional do HAD e do EM. Extraído de (cds.cern.ch)}
	\label{fig:2T13}
\end{figure}

Quando partículas 'excitam' as telhas, este material cintilador produz luz, que é transmitida por fibras ópticas até \ac{TFM}. Este instrumento converte a luz em sinal elétrico, que é lido pela eletrônica do TileCal.

\subsection{O Detector de Múons}\label{subsec:cam_muon}

A camada mais externa do detector ATLAS é a câmara de múons. Idealmente, essas partículas são as únicas, detectáveis, capazes de atravessar os calorímetros. O espectrômetro de múons \cite{atlas2010commissioning} circunda o calorímetro e mede as trajetórias dessas partículas, sendo, assim, capaz de medir o seu momento junto com o ID. Esses traços sempre são normais a componente principal do campo eletromagnético, o que torna o resolução do momento transverso rudemente independente de $\eta$.

O sistema de detecção de múons é constituído por milhares de sensores de partículas carregadas, colocados em um campo magnético, produzido por grandes bobinas toroidais supercondutoras. Na Figura~\ref{fig:2T14}, é mostrado o modelo computacional do Detector de Múons.

\begin{figure}[h!]
	\centering
	\includegraphics[width=12cm]{./textuais/atlas/figuras/Camara_Muon.png}\\
	\caption{Modelo computacional da Câmara de Múons do detector ATLAS. Extraído de (cds.cern.ch).}
	\label{fig:2T14}
\end{figure}


\subsection{Sistema de Filtragem do ATLAS}\label{sec:sis_fil}

No intuito de alcançar novas descobertas, utilizando-se de eventos raros, o experimento efetua colisões com uma taxa muito alta de operação, gerando um conjunto muito grande de informações, onde grande parte desses eventos pode ser descartada, a fim de evitar o armazenamento de dados não relevantes ou mesmo já bem explorados \cite{1352047}. Nesse contexto, faz-se necessário um Sistema de Filtragem \emph{Online} que seja capaz de separar os eventos considerados importantes e armazená-los para um análise posterior mediada por algoritmos mais complexos e criteriosos.

Como a Figura~\ref{fig:2T16} apresenta, o sistema de filtragem \emph{online} foi desenvolvido em 3 níveis consecutivos que, juntos, reduzem a taxa de eventos de 40 MHz para 200 Hz \cite{elsing2003configuration}. Esses níveis são o \ac{L1}, \ac{L2} e \ac{EF}, respectivamente.

\begin{itemize}
  \item O primeiro nível de \emph{trigger} é feito em hardware, uma vez que precisa trabalhar com uma latência de $\sim$2$\mu$s. Esse sistema recebe os sinais dos calorímetros e da câmara de múons, separando os conjuntos de eventos que ficaram dentro do limiar de corte estabelecido, chamados de \ac{RoI}, reduzindo a taxa de eventos para $75 kHz$ \cite{gabaldon2012performance};
  \item O segundo nível de filtragem tem sua implementação baseada em softwares operando em uma rede de computadores e sua principal característica é observar as \ac{RoI} pré-definidas pelo L1. Este tem a capacidade de reduzir para $\sim$2 $kHz$ a taxa de eventos;
  \item Já o último nível de filtragem do ATLAS reduz essa taxa para 200 Hz, trabalhando com uma granularidade maior, re-selecionando as informações transmitidas pelo L2.
\end{itemize}


\begin{figure}[h!]
	\centering
	\includegraphics[width=9cm]{./textuais/atlas/figuras/sistema_de_filtragem.pdf}\\
	\caption{Fluxograma do sistema de Trigger Online do ATLAS. Extraído de \cite{dos2006sistema}.}
	\label{fig:2T16}
\end{figure}

%O sistema de filtragem \emph{online} tem o intuito de reduzir a taxa de eventos sem prejudicar a visualização de um possível canal físico relevante, portanto, esta ferramenta deve operar com uma eficiência de detecção alta. Entretanto, isso proporciona um aumento no falso alarme. Consequentemente, a saída deste sistema retorna eventos de interesse juntamente com ruídos de fundo \cite{torres2010sistema}.

Como foi dito na Seção~\ref{sec:ATLAS}, o ATLAS é um detector de uso geral; portanto, os dados armazenados são utilizados em diferentes estudos, que são realizados por filtragem \emph{offline}. Uma vez que este sistema não tem como fator determinante o tempo de processamento, algoritmos mais bem elaborados e específicos para cada tipo de estudo podem ser empregados, possibilitando, assim, uma identificação mais robusta das partículas. O ambiente \emph{offline} é também propício para implementação e testes de novos algoritmos que podem, eventualmente, ser aplicados futuramente no sistema de filtragem \emph{online} do ATLAS.

\chapter{Conclusão} \label{cap:conclusao}
A discretização contínua de dados representa uma importante tarefa de pré-processamento no contexto de estimação de dados. Até agora, muito trabalho foi feito para melhorar a discretização, a fim de reduzir as informações redundantes ou desconectadas.

Este trabalho abordou o problema de discretização no cenário de estimação de densidades, fazendo uma avaliação cuidadosa do assunto na literatura, propondo quatro diferentes métodos de discretização e comparando-os com o método \textit{Linspace}, amplamente utilizado na literatura. Em particular, o início e o final da região da variável aleatória devem ser definidos antes de aplicar a discretização, cujo desempenho é altamente sensível a esses parâmetros. Nesse contexto, pode ser importante procurar métodos mais resilientes.

Os resultados apresentados mostraram que os métodos propostos são mais resilientes aos outliers e destacaram as vantagens e desvantagens de cada método, mostrando claramente que o processo de discretização é de fundamental importância para minimizar o erro de estimação. Além disso, este trabalho fornece uma base sólida de conhecimento sobre esta questão, tornando este estudo uma possível base para futuras investigações aprofundadas.
\section{Próximos Passos}
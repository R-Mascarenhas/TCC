\chapter{Conclusão} \label{cap:conclusao}
\vspace{-2cm}
A discretização contínua de dados representa uma importante tarefa de pré-processamento no contexto de estimação de dados. Até agora, muito trabalho foi feito para melhorar a discretização, a fim de reduzir as informações redundantes ou desconectadas.

Este trabalho abordou o problema de discretização no cenário de estimação de densidades, fazendo uma avaliação cuidadosa do assunto na literatura, propondo quatro diferentes métodos de discretização e comparando-os com o método \textit{Linspace}, amplamente utilizado na literatura. Em particular, o início e o final da região da variável aleatória devem ser definidos antes de aplicar a discretização, cujo desempenho é altamente sensível a esses parâmetros. Nesse contexto, pode ser importante procurar métodos mais resilientes.

Vale resaltar que estes métodos se comportaram de maneiras diferentes quando submetidos a funções analíticas ou com dados gerados, sendo que este último se agrava em distribuições assimétricas ou com caldas muito longas devido ao ruído que é inserido quando se deriva uma função discreta que já é ruidosa, e, curiosamente, este ruído contribuiu para que estes métodos (\ac{iPDF1} e \ac{iPDF2}) apresentassem resultados melhores que os esperados, fazendo-os uma boa opção de discretização apesar de seu algorítimo ser um pouco mais complexo de se implementar se comparado aos métodos \textit{Linspace} e \ac{CDFm}.

\section{Próximos Passos}
Ao final desse estudo, fica claro que existem possibilidades de melhorias nos métodos apresentados, tendo a necessidade de uma análise mais profunda para outras distribuições além das Normais e Lognormais além de se buscar outros métodos complementares aos já abordados ou até mesmo algum método que junte dois ou mais métodos, fazendo com que se pegue a melhor região para cada um deles.
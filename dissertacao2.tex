\documentclass[a4paper,12pt,normaltoc,capchap,capsec,pagestart=firstchapter,tocpage=plain,sumariocompleto]{abnt_ufjf}


%%%%%%%%%%%%%%%%%%%%%%%%%%%%%%%%%
%                               %
%     Configura\c{c}\~{a}o gerais       %
%                               %
%%%%%%%%%%%%%%%%%%%%%%%%%%%%%%%%%


%pacotes idioma
\usepackage[brazil]{babel}          % suporte para os termos na l\'{\i}ngua portuguesa
\usepackage[T1]{fontenc}          % l\^{e} a codifica\c{c}\~{a}o de fonte T1
\usepackage[utf8]{inputenc}         % suporte para caracteres especiais
\usepackage{times}                  % fonte "Times"
\usepackage{ae}                     % fonte "Almost European"
%
%figuras
\usepackage[dvips]{graphicx}        % para inclus\~{a}o de figuras.
\usepackage{subfigure}              % inclus\~{a}o de subfigura.
\usepackage{color}
\usepackage{float}
%\usepackage{rotating}               %rotacionar a figura e a legenda
\usepackage{psfrag}


% pacote matem\'{a}tico
\usepackage{amsmath}                %pacote matem\'{a}tico da American Mathematical Society.
\usepackage{amsfonts}
\usepackage{amssymb}
\usepackage{amstext}
\usepackage{mathrsfs}               %pacote para fontes usadas em transformadas.
%\usepackage{icomma}                 %nota\c{c}\~{a}o decimal com virgula.
\usepackage{bm}                     %bold math

\DeclareMathOperator{\sen}{sen}     %declara operador seno.
\DeclareMathOperator{\inteiro}{int} %declara operador inteiro.
\DeclareMathOperator{\real}{Re}     %declara operador real.
\DeclareMathOperator{\imag}{Im}     %declara operador imaginario.
\DeclareMathOperator{\sign}{sinal}   %sinal

% formata\c{c}\~{a}o
\usepackage{indentfirst}            % indenta os primeiros par\'{a}grafos.
\usepackage[printonlyused]{acronym} % pacote para produzir acronimos
%\usepackage{a4wide}
\usepackage{geometry}
\usepackage{setspace}
\usepackage{enumerate}
\usepackage{url}
\usepackage{pifont}

%tabelas
\usepackage{multirow}

\geometry{verbose, a4paper, tmargin=3.0cm, bmargin=2.0cm, lmargin=3.0cm, rmargin=2.5cm}

% bibliografia
\usepackage[alf]{abntcite}          %chamada de referencia alfabetica
%\usepackage[num]{abntcite}          %chamada de referencia num\'{e}rica

%%%%%%%%%%%%%%%%%%%%%%%%%%%%%%%%%
%                               %
%     Dados da disserta\c{c}\~{a}o      %
%                               %
%%%%%%%%%%%%%%%%%%%%%%%%%%%%%%%%%

%\newcommand{\Titulo}{Modelagem e Controle de Conversores Fonte de Tens\~{a}o Utilizados em Sistemas de Gera\c{c}\~{a}o Fotovoltaicos Conectados \`{a} Rede El\'{e}trica de Distribui\c{c}\~{a}o}
\newcommand{\TITULO}{Proposta de aplicação do método de Verossimilhança nas variáveis de Ringer e Traço para identificação de elétrons de um detector de fina segmentação}
\newcommand{\Autor}{Igor Abritta Costa}
\newcommand{\Orientador}{Rafael Antunes Nóbrega, D.Sc.}
%\newcommand{\Coorientador}{José Manoel Seixas, D.Sc.}
\newcommand{\Dia}{22 }
\newcommand{\Mes}{Maio }
\newcommand{\Ano}{2018}

%%%%%%%%%%%%%%%%%%%%%%%%%%%%%%%%%
%                               %
%        Dados da banca         %
%                               %
%%%%%%%%%%%%%%%%%%%%%%%%%%%%%%%%%

\newcommand{\PrimeiroExaminador}{Prof. \Orientador}
\newcommand{\InstituicaodoPrimeiroExaminador}{Universidade Federal de Juiz de Fora, UFJF}
\newcommand{\SegundoExaminador}{Prof. xx}
\newcommand{\InstituicaodoSegundoExaminador}{Universidade Federal do Rio de Janeiro, UFRJ}
\newcommand{\TerceiroExaminador}{Prof. x, D.Sc.}
\newcommand{\InstituicaodoTerceiroExaminador}{Universidade Federal de Juiz de Fora, UFJF}
\newcommand{\QuartoExaminador}{Prof. x, D.Sc.}
\newcommand{\InstituicaodoQuartoExaminador}{Universidade Federal do Rio de Janeiro, UFRJ}


%\hyphenation{li-ne-ar mo-de-la-gem}


\begin{document}


%%%%%%%%%%%%%%%%%%%%%%%%%%%%%%%%%
%                               %
%         Pr\'{e} textuais          %
%                               %
%%%%%%%%%%%%%%%%%%%%%%%%%%%%%%%%%

\thispagestyle{empty}
\begin{center}

\includegraphics[width=3.0cm]{./logos/ufjf_logo}\\
\medskip
Universidade Federal de  Juiz de Fora\\
Engenharia\\
Bacharelado em Engenharia Elétrica

\vfill

\Autor\\

\vfill


\TITULO\\

\vfill

Trabalho de Conclusão de Curso\\

\vfill

Juiz de Fora\\
\Ano\\

\end{center}
\newpage


\thispagestyle{empty}

\begin{center}

\Autor\\

\vfill

\TITULO

\vfill

\begin{flushright}
    \begin{tabular}{p{8.0cm}}
    Qualifica\c{c}\~{a}o apresentada ao Programa de P\'{o}s--Gradua\c{c}\~{a}o em Engenharia El\'{e}trica, \'{a}rea de concentra\c{c}\~{a}o: Sistemas Eletr\^{o}nicos, da Faculdade de Engenharia da Universidade Federal de Juiz de Fora como requisito parcial para obten\c{c}\~{a}o do grau de Doutor.
    \end{tabular}
\end{flushright}

\vfill

\begin{flushleft}
    \begin{tabular}{rl}
    Orientadores:   & Prof. \Orientador\\
                    %& Prof. \Coorientador\\ %descomentar essa linha quando houver coorientador
    \end{tabular}
\end{flushleft}

\vfill

Juiz de Fora\\
\Ano\\

\end{center}
\newpage


%\

\vspace{6.0cm}

\begin{center}
\begin{tabular}{|p{12cm}|}
\hline\\
\hspace{1.0cm}\parbox{10.0cm}{

        \vspace{0.5cm}

        Costa, Rafael Mascarenhas

        \medskip

        \hspace{0.5cm}\TITULO / \Autor. - 2018.

        \hspace{0.5cm}107 f. : il.

        \medskip

        \hspace{0.5cm}Disserta\c{c}\~{a}o (Trabalho de Conclusão de Curso) - Universidade Federal de Juiz de Fora, 2018

        \medskip

        \hspace{0.5cm}1. Identificação de Elétrons. 2. \emph{Likelihood}. 3. KDE Multivariado I. T\'{\i}tulo.

        \medskip

        \hspace{6.0cm}CDU 621.3.0

        \vspace{1.0cm}
        }\\
\hline
\end{tabular}
\end{center}

\newpage 

\thispagestyle{empty}

\begin{center}

\Autor

\vfill

\TITULO

\vfill

\end{center}

\begin{flushright}
    \begin{tabular}{p{8.0cm}}
    Qualifica\c{c}\~{a}o apresentada ao Programa de P\'{o}s--Gradua\c{c}\~{a}o em Engenharia El\'{e}trica, \'{a}rea de concentra\c{c}\~{a}o: Sistemas Eletr\^{o}nicos, da Faculdade de Engenharia da Universidade Federal de Juiz de Fora como requisito parcial para obten\c{c}\~{a}o da graduação.
    \end{tabular}
\end{flushright}

\vspace{1.0cm}

\noindent Aprovada em \Dia de \Mes de \Ano.\\

\begin{center}

BANCA EXAMINADORA:\\

\vfill

\begin{tabular}{c}
\\
\\
\hline
\textbf{\PrimeiroExaminador}\\
\InstituicaodoPrimeiroExaminador\\
Orientador\\
\\
\\
\hline
\textbf{\SegundoExaminador}\\
\InstituicaodoSegundoExaminador\\
%Orientador\\ % Descomentar essa linha abaixo quando houver coorientador
\\
\\
\hline
\textbf{\TerceiroExaminador}\\
\InstituicaodoTerceiroExaminador\\
\\
\\
% \hline
% \textbf{\QuartoExaminador}\\
% \InstituicaodoQuartoExaminador\\
%
% Descomentar as linhas abaixo se houver quinto e sexto examinadores
%
%\\
%\\
%\hline
%\textbf{\QuintoExaminador}\\
%\InstituicaodoQuintoExaminador\\
%\end{tabular}
%\\
%\\
%\hline
%\textbf{\SextoExaminador}\\
%\InstituicaodoSextoExaminador\\
\end{tabular}

\end{center}

\newpage


%%%%%%%%%%%%%%%%%%%%%%%%%%%%%%%%%%%%%%%%%%%%%%%%%%%%%%%%%%%%%%%%%%%%%%%%%%%%%%%%%%%%%%%%%%%%%%%%%%%%%%%%%%%
%
% Dedicat\'{o}ria
%
%%%%%%%%%%%%%%%%%%%%%%%%%%%%%%%%%%%%%%%%%%%%%%%%%%%%%%%%%%%%%%%%%%%%%%%%%%%%%%%%%%%%%%%%%%%%%%%%%%%%%%%%%%
\

\vfill

\begin{flushright}
\hfill \textit{Aos meus pais, meus irmãos, minha namorada, aos meus familiares, aos meus amigos.}
\end{flushright}
\vspace*{1cm}

\clearpage




%%%%%%%%%%%%%%%%%%%%%%%%%%%%%%%%%%%%%%%%%%%%%%%%%%%%%%%%%%%%%%%%%%%%%%%%%%%%%%%%%%%%%%%%%%%%%%%%%%%%%%%%%%%
%
% Agradecimentos
%
%%%%%%%%%%%%%%%%%%%%%%%%%%%%%%%%%%%%%%%%%%%%%%%%%%%%%%%%%%%%%%%%%%%%%%%%%%%%%%%%%%%%%%%%%%%%%%%%%%%%%%%%%%

\chapter*{Agradecimentos}
\vspace{-1.5cm}

Agradeço primeiramente aos meus pais, por me darem plenas condições de estudo, além de todo apoio moral nessa jornada, aos meus amigos e colegas pelas risadas e choros durante a vida acadêmica. 

Aos meus nobres professores pelos ensinamentos que contribuiram com meu crescimento profissional e pessoal. Em especial ao meu orientador, Rafael e coorientador, Igor, por acreditarem em mim e me guiarem na vida acadêmica. Meus sinceros agradecimentos. %pela paciência e por acreditar no meu trabalho, além de todo o povo do chiqueirinho pelas risadas e bons papos

Aos companheiros do NIPS e amigos de Laboratório. Pelas conversas, ajudas e bom humor. É sempre bom ter alguém para compartilhar alegrias e desespero.

Aos meus amigos do ramo IEEE por toda motivação e crescimento e por me mostrar que estou no curso certo.

Finalmente, agradeço à Universidade Federal de Juiz de Fora e à Faculdade de Engenharia por todo o suporte e pelas ferramentas necessárias ao desenvolvimento deste trabalho.

%%%%%%%%%%%%%%%%%%%%%%%%%%%%%%%%%%%%%%%%%%%%%%%%%%%%%%%%%%%%%%%%%%%%%%%%%%%%%%%%%%%%%%%%%%%%%%%%%%%%%%%%%%%
%
% Ep\'{\i}grafe
%
%%%%%%%%%%%%%%%%%%%%%%%%%%%%%%%%%%%%%%%%%%%%%%%%%%%%%%%%%%%%%%%%%%%%%%%%%%%%%%%%%%%%%%%%%%%%%%%%%%%%%%%%%%

\

\vfill

\begin{flushright}
\begin{minipage}{.5\textwidth}
	\textit{Não vemos as coisas como elas são, mas como nós somos.}\\
    \begin{flushright}
	  Anaïs Nin
    \end{flushright}
\end{minipage}
\end{flushright}

\vspace*{1cm}
\newpage


%%%%%%%%%%%%%%%%%%%%%%%%%%%%%%%%%%%%%%%%%%%%%%%%%%%%%%%%%%%%%%%%%%%%%%%%%%%%%%%%%%%%%%%%%%%%%%%%%%%%%%%%%%
%
% Resumo
%
%%%%%%%%%%%%%%%%%%%%%%%%%%%%%%%%%%%%%%%%%%%%%%%%%%%%%%%%%%%%%%%%%%%%%%%%%%%%%%%%%%%%%%%%%%%%%%%%%%%%%%%%%%
\chapter*{Resumo}


\noindent 
\vspace{0.5cm}

\noindent Palavras-chave:  \\

\newpage




%%%%%%%%%%%%%%%%%%%%%%%%%%%%%%%%%%%%%%%%%%%%%%%%%%%%%%%%%%%%%%%%%%%%%%%%%%%%%%%%%%%%%%%%%%%%%%%%%%%%%%%%%%
%
% Resumo
%
%%%%%%%%%%%%%%%%%%%%%%%%%%%%%%%%%%%%%%%%%%%%%%%%%%%%%%%%%%%%%%%%%%%%%%%%%%%%%%%%%%%%%%%%%%%%%%%%%%%%%%%%%%
\chapter*{Abstract}


\noindent The particle identification has fundamental importance to the high-energy physics experiments developed around the world. In this environment of particle physics, the particles occurrence probability relevant to the proposed studies is very low in relation to the particles that form the background noise, requiring algorithms with efficiency indices of interest signals detection and background noise rejection each time best. Many methods of identifying particles using the likelihood technique, which uses the probability distribution of variables to create a discriminator. Therefore, to guarantee the performance of the classification, a good quality of estimation of the distributions in question is necessary, being these commonly different from the distributions known and parameterized in the literature. In this thesis, the methods applied to the estimation of non-parametric density will be reviewed and possible optimizations will be evaluated from the data produced by one of the largest experiments of CERN, the ATLAS. Concentrated in the offline context, the paper reproduces the likelihood-based method and proposes some improvements with the use of more optimized processing algorithms, robust statistics and discretization techniques different from those used in the literature. In addition, this work proposes to expand the tools used in the univariate estimation for the multivariate density estimation environment, known as MKDE (Multivariate Kernel Density Estimation), which may be able to mitigate the inserted error in the consideration of independence between the discriminant variables inserted by the method of Likelihood currently in use by several experiments of this type. Initially, this work proposes to implement the likelihood method based on the estimation of univariate densities used in the reconstruction of the joint density of the discriminant variables and to study the impact of possible parameters related to the implementation of the algorithm for estimating univariate densities. In a second step, the implementation of MKDE is inserted through a direct comparison with the univariate method.


\vspace{0.5cm}

\noindent Keywords: Density estimation, Likelihood, FastKDE, Discretization. \\

\newpage


\renewcommand{\listfigurename}{Lista de Ilustra\c{c}\~{o}es}
\listoffigures

\listoftables

%\begin{siglas} %%ALTERAR OS EXEMPLOS ABAIXO, CONFORME A NECESSIDADE
\chapter*{LISTA DE ABREVIATURAS E SIGLAS}
\vspace{-1.5cm}
	\begin{acronym}
		\acro{AMISE}{Erro Quadrático Integrado Assintótico Médio (do inglês, \emph{Asymptotic Mean Integrated Squared Error}}
		\acro{AMSE}{Erro Quadrático Médio Assintótico (do inglês, \emph{Asymptotic Mean Squared Error}}
		\acro{ATLAS}{Dispositivo Instrumental Toroidal para o LHC (do inglês, \textit{A Toroidal LHC Apparatus})}
		\acro{BCV}{Validação cruzada tendenciosa (do inglês, \emph{Biased Cross-Validation}}
		\acro{CDF}{Função Distribuição Cumulativa (do inglês, \textit{Cumulative Distribution Function})}
		\acro{CDFm}{Método CDF (do inglês, \textit{CDF method})}
		\acro{CERN}{Centro Europeu de Pesquisa Nuclear, (do francês, \textit{Conseil Européen pour la Recherche Nucléaire})}
		\acro{CMS}{Solenoide de Múon Compacto (do inglês, \emph{Compact Muon Solenoid})}
		\acro{FD}{Estimador Freedman–Diaconi}
		\acro{IAE}{Erro Absoluto Integrado (do inglês,\emph{Integrated Absolute Error}}
		\acro{iPDF1}{Integral da distribuição da primeira derivada da PDF}
		\acro{iPDF2}{Integral da distribuição da segunda derivada da PDF}
		\acro{ISE}{Erro Quadrado Integrado (do inglês,\emph{Integrated Squared Error}}
		\acro{KDE}{Estimação de Densidade de Núcleo, (do inglês, \textit{Kernel Density Estimation})}
		\acro{LHC}{Grande Colisor de Hádrons (do inglês, \textit{Large Hadron Collider})}
		\acro{LSCV}{Validação Cruzada pelo Mínimo Quadrado (do inglês, \emph{Least-Square Cross-Validation}}
		\acro{MISE}{Erro Quadrático Integrado Médio (do inglês, \emph{Mean Integrated Squared Error}}
		\acro{MKDE}{Estimativa Multivariada de Densidade de Kernel (do inglês, \emph{Multivariate Kernel Density Estimation)}}
		\acro{MSE}{Erro Quadrático Médio (do inglês, \emph{Mean Squared Error}}
		\acro{MVA}{Análise Multivariada, (do inglês, \emph{Multivariate Analysis})}
		\acro{PDF}{Função Densidade de Probabilidade (do inglês, \textit{Probability Density Function})}
		\acro{PDFm}{Método PDF (do inglês, \textit{PDF method})}	
		\acro{RoI}{Região de Interesse (do inglês, \textit{Region of Interest})}
	\end{acronym}

%\end{siglas}

\tableofcontents

%%%%%%%%%%%%%%%%%%%%%%%%%%%%%%%%%
%                               %
%     Corpo da disserta\c{c}\~{a}o      %
%                               %
%%%%%%%%%%%%%%%%%%%%%%%%%%%%%%%%%

%\chapter{INTRODUÇÃO}

%Atualmente, devido à complexidade crescente dos experimentos científicos, há um aumento no uso de estimadores não paramétricos de densidade para se obter a PDF das distribuições que não podem ser descritas por funções paramétricas conhecidas. Existem vários trabalhos em andamento relatando este tipo de estimação em diferentes áreas do conhecimento como Engenharia, Física, Biologia, Economia, Ecologia, Geologia, dentre outras. Este trabalho tem como plano de fundo o alto nível de performance demandada de algoritmos de estimação e classificação em experimentos de física de partículas, como em \cite{aad2008atlas}.

%Geralmente, neste tipo de experimento, os algoritmos estão relacionados à viabilidade dos estudos propostos pelos físicos, uma vez que, geralmente, os eventos de interesse, chamados de sinais, são extremamente raros e contaminados com um alto nível de ruído de fundo. Contudo, é necessário construir algorítimos capazes de identificar eventos com a menor taxa de erro possível, o que implicitamente significa delimitar e estimar características probabilísticas de todos os eventos conhecidos da melhor maneira possível.

%Atualmente, diversos algoritmos de classificação no contexto de \textit{big data} usam KDE não paramétrico \cite{scott2015multivariate}, Os experimentos do Centro Europeu de Pesquisa nuclear, (do frânces, \textit{Conseil Européen pour la Recherche Nucléaire}) (CERN) são exemplos disso no caso de algoritmo de identificação de elétrons aplicado ao experimento ATLAS \cite{atlas2014electron}. Tais algoritmos necessitam não apenas de uma performance ótima, mas também de menos poder computacional.


\label{cap:intro}

A crescente evolução tecnológica vem possibilitando o desenvolvimento de muitas áreas do conhecimento, sendo uma delas a engenharia elétrica, mais precisamente a análise multivariada \cite{vicini2005analise} que torna-se cada vez mais uma ferramenta importante para a solução de problemas ligados à estimação de densidades e seleção de eventos, tanto no ambiente industrial quanto em laboratórios de pesquisa. Entretanto, tais problemas podem ocorrem em outras áreas do conhecimento, sendo assim, o esforço em prol da otimização dessas ferramentas de maneira multidisciplinar é de grande interesse.

Nas últimas décadas, a importância de uma modelagem estocástica por \ac{PDF}, utilizando-se de métodos não paramétricos teve um crescimento considerável devido ao fato de que vários experimentos geradores de enorme quantidade de dados foram iniciados.  Os experimentos ligados ao \ac{LHC} representam alguns deles. Desde a criação do \ac{CERN}, físicos e engenheiros de diferentes países têm trabalhado em conjunto para investigar questões referentes ao estado da arte da ciência fundamental relacionada à física de altas energias, usando instrumentos científicos complexos para estudar os constituintes básicos da matéria e suas interações. No complexo principal do \ac{CERN}, o \ac{LHC}, prótons são colocados em um acelerador que os faz colidir quase à velocidade da luz. Este processo permite estudar como as partículas interagem e fornece uma visão das leis fundamentais da natureza \cite{cernwebabout}.


Atualmente, a física experimental de altas energias é um ramo da ciência em progressiva expansão e pode ser considerada um dos campos científicos mais exigentes em termos de processamento de sinal, esse fato é explicado devido aos eventos de interesse serem raros e contaminados com alto nível de ruído de fundo, demandando sistemas cada vez mais otimizados no que diz respeito a tempo de processamento, eficiência de detecção de sinal e rejeição de ruído.

Com o objetivo de observar os subprodutos dessas colisões, é necessário usar detectores; basicamente, sensores que, trabalhando em conjunto, são capazes de medir algumas características dos subprodutos das colisões e transformá-los em sinais elétricos que podem ser armazenados e utilizados em estudos relacionados a física de altas energias.

Em geral, para problemas cujas variáveis podem ser modeladas, a estimação das mesmas se torna paramétrica. No entanto, é muito importante enfatizar que, devido à complexidade do problema, suas variáveis podem não ser descritas com as funções de densidade de probabilidade conhecidas na literatura. Sendo assim, a aplicação de métodos não paramétricos se espalhou consideravelmente nos últimos anos devido às ferramentas recentemente desenvolvidas para análise estatística. Tais métodos  fornecem um caminho alternativo a estimação paramétrica e possibilita o estudo de grandes quantidades de dados, essa linha de pesquisa torna-se objeto significativo de estudo, uma vez que contempla pesquisas teóricas e práticas com relação direta a temas como regressão, discriminação e reconhecimento de padrões. 

%Neste contexto, abre-se então a possibilidade de buscar uma combinação fundamentada entre estimação não paramétrica e processamento estatístico multivariado, no intuito de garantir uma resposta eficaz, robusta e viável.


 Neste contexto, o presente trabalho visa avaliar os erros inseridos pelo processo de discretização propondo diferentes métodos e olhando diretamente à sua performance de estimação, considerando as interpolações pelo vizinho mais próximo e linear. O impacto pelos pontos longe da região de alta probabilidade também são avaliados uma vez que é um problema comum na estimação de \ac{PDF}.

\section{Motivação}

%Neste trabalho, serão propostos novos métodos de discretização a fim de reduzir os erros de estimação sem demandar de mais poder computacional para tal, tentando garantir uma quantidade de pontos adequadas a cada região de interesse.
Na última década muitos trabalhos relacionados ao tema de otimização da estimação de densidade não paramétrica, tanto numérica \cite{schindler2012bandwidth} quanto computacional \cite{gramacki2017nonparametric}, foram publicados, bem como sobre os temas de discretização, estatística robusta e medidas de distância, mostrando que são temas que mesmo sendo discutidos há décadas ainda estão sendo utilizados, explorados e em desenvolvimento. Além disso, experimentos complexos de Big Data, como os do LHC, têm aplicado análises usando estimação de densidade em conjunto com técnicas de verossimilhança empregados em problemas de identificação de partículas obtendo resultados relevantes, mesmo utilizando uma simplificação da formulação matemática desse método, assumindo independência entre variáveis. Portanto, abre-se a possibilidade de contribuir nessa área no que diz respeito a otimização da estimação de densidades e suas nuances, usando como base de desenvolvimento um sistema altamente complexo, com grande número de variáveis e distribuições com características bastante distintas, como ocorre com os experimentos do \ac{LHC}. 

\section{Estrutura do Trabalho}
Este documento está organizado da seguinte maneira: XXXXXXXX


%\chapter{Física de Altas Energias e o CERN}\label{cap:LHC}
	
Desde os anos setenta, os físicos de partículas têm descrito a estrutura fundamental da matéria usando uma elegante série de equações denominado Modelo Padrão. O modelo tenta descrever tudo que pode ser observado no universo, a partir de alguns blocos básicos chamados partículas fundamentais \cite{cernphisics}.

Portanto, para a comprovação experimental dessa teoria, equipamentos que são capazes de colidir partículas em altas energias foram construídos, recriando um ambiente onde é possível observar, mais profundamente, as partículas fundamentais e seus processos de interação. Desde 1911, quando o primeiro acelerador de partículas foi criado pelo físico britânico Ernest Rutherford \cite{aceleradores2015}, aceleradores vem sendo construídos com energias de colisão cada vez maiores. Em 2008, o \ac{CERN} inaugurou o maior acelerador de partículas do mundo, o \ac{LHC}, projetado para colidir prótons a uma energia de centro de massa de 14 TeV.
	
Esta seção é dedicada a ambientar e dar uma visão geral sobre qual é o objeto de estudo e onde este estudo é realizado, ou seja, será feita uma breve explicação sobre o Modelo Padrão, o CERN, o experimento LHC e o detector ATLAS.


\section{Modelo Padrão}

O Modelo Padrão das partículas elementares, não é exatamente um modelo, mas sim uma teoria muito bem fundamentada, considerada a melhor teoria sobre a natureza da matéria por muitos físicos \cite{moreira2009modelo}.

De acordo com o Modelo Padrão, todas  as  partículas  podem  ser  classificadas  em  Bósons  e  Férmions, sendo que os primeiros não obedecem o Princípio de Exclusão de Pauli, que é um princípio da mecânica quântica, que afirma que dois férmions idênticos não podem ocupar o mesmo estado quântico simultaneamente. Este modelo também descreve os mecanismos de interação regidos pelas forças: eletromagnética, fraca e forte; a única força não abrangida por esta teoria é a força gravitacional. \cite{perkins2000introduction}

As partículas constituintes da matéria podem ser divididas e nomeadas como segue:
\begin{itemize}
  \item Férmions: partículas que constituem a matéria e são subdivididas em léptons e quarks.
  \item Léptons: elétron, múon, tau e seus neutrinos e suas antipartículas.
  \item Quarks são: up, down, charm, strange, top e bottom e suas antipartículas.
\end{itemize}

%\begin{figure}[!h]
%  \centering
%  \includegraphics[width=10cm]{./textuais/experimento/figuras/fig3.png}\\
%  \caption{Modelo Padrão. Extraído de \cite{fehling2011standard}.}
%  \label{fig:2T01}
%\end{figure}

As partículas transportadoras de força que mediam as interações entre partículas e são: glúon(força forte), fóton(força eletromagnética), bósons W e Z(força fraca) e o bóson de Higgs(responsável pela existência de massa inercial).

Um resumo das informações das partículas do Modelo Padrão é apresentado na Figura~\ref{fig:2T02}.

\begin{figure}[!h]
  \centering
  \includegraphics[width=12cm]{./textuais/experimento/figuras/fig3.png}\\
  \caption{Modelo Padrão. Extraído de \cite{modelopadrao}.}
  \label{fig:2T02}
\end{figure}

\section{CERN}

No CERN, físicos e engenheiros trabalham em conjunto com o objetivo de investigar a estrutura fundamental do universo. Fundada em 1954, o laboratório CERN foi construído na fronteira franco-suíça, em Genebra. Ele foi um dos primeiros empreendimentos conjuntos da Europa e tem agora 21 Estados membros \cite{cernwebabout}.

O principal foco desta organização é a física de partículas que abrange estudos como: composição, raios cósmicos, matéria escura, dimensões extras, grávitons, minúsculos buracos negros, íons pesados, plasma quark-glúon, entre outros \cite{cernphisics}.

A necessidade de comprovar as teorias e estudar as partículas de maneira mais profunda tornou a construção do acelerador de partículas \ac{LHC} imprescindível. Nele, feixes de prótons são acelerados em direções opostas até atingirem altas energias e colidirem uns com os outros. A Seção~\ref{sec:LHC} abordará melhor a estrutura deste aparato.

Com o intuito de 'ler' e armazenar as informações geradas nas colisões dentro do acelerador, faz-se necessária a utilização de detectores, no local exato das colisões entre os feixes. Atualmente. o LHC conta com alguns detectores como: \ac{ALICE}, \ac{ATLAS}, \ac{CMS}, \ac{LHCb}. A Figura~\ref{fig:2T04} mostra a localização dos detectores (ALICE, ATLAS, CMS e LHCb) no LHC.

\begin{figure}[!h]
  \centering
  \includegraphics[width=12cm]{./textuais/experimento/figuras/fig1.png}\\
  \caption{Uma visão geral do experimento LHC. Extraído de CERN \cite{cernwebabout}.}
  \label{fig:2T04}
\end{figure}

Na Seção~\ref{sec:ATLAS}, o detector de ATLAS será melhor detalhado, uma vez que os dados utilizados nessa dissertação foram gerados por esse detector.

\section{LHC}\label{sec:LHC}

O LHC,, Figura~\ref{fig:2T03}, tem cerca de 27 km de circunferência. Ele acelera prótons ou íons que viajam em direções opostas, e são colocados para colidir \cite{lefevre2009lhc}.

Um acelerador só pode acelerar certos tipos de partícula: em primeiro lugar, esses elementos precisam ter carga, uma vez que os feixes são manipulados por dispositivos eletromagnéticos que só podem influenciar as partículas carregadas; e, em segundo lugar, exceto em casos especiais, estas não podem decair. Isso limita o número de partículas que podem ser acelerados para elétrons, pósitrons, prótons e íons. É necessário acrescentar que em um acelerador circular, como o LHC, partículas pesadas, como prótons, têm uma perda de energia, através de radiação síncrotron, muito menor que partículas leves, como elétrons. Portanto, para obter colisões com energias muito elevadas, o LHC faz uso de prótons.

%
\begin{figure}[!h]
  \centering
  \includegraphics[width=12cm]{./textuais/experimento/figuras/fig2.png}\\
  \caption{O LHC é o maior e mais poderoso acelerador de partículas do mundo. Extraído de \cite{cernwebLHC}.}
  \label{fig:2T03}
\end{figure}
%

O complexo de aceleradores do CERN é uma sucessão de mecanismos que aceleram partículas a energias cada vez maiores. Cada um desses instrumentos aumenta a energia de um feixe de partículas, antes de injetar o feixe no LHC, propriamente dito.

Com a ajuda de um campo eletromagnético, os prótons dos átomos de hidrogênio são separados dos elétrons. Estes prótons, primeiramente, são acelerados a energia de 50 MeV pelo \ac{LINAC 2}. Esse feixe é então injetado no \ac{PSB}, que o leva a energia de 1,4 GeV, seguido pelo \ac{PS}, que o impulsiona a 25 GeV. Esses prótons são enviados para o \ac{SPS}, onde eles são acelerados para 450 GeV. No LHC, o último elemento nesta cadeia, feixes de partículas são aceleradas até a energia recorde de 7 TeV por feixe, nominal de operação, com que colidem \cite{cernwebAccelerator}.

%
No ano de 2015, depois de passar por manutenção e modificações, o LHC retomou as atividades preparado para atingir energia de colisão no centro de massa de 13 TeV, aproximadamente o dobro da energia que vinha trabalhando. Este \emph{upgrade} tem o intuito de alcançar novos resultados e descobertas, uma vez que funcionará a uma energia nunca antes alcançada.

\section{ATLAS}\label{sec:ATLAS}

Com 46 metros de comprimento, 25 metros de altura e 25 metros de largura, e 7 mil toneladas, o detector ATLAS, mostrado na Figura~\ref{fig:2T05}, é o maior detector de partículas já construído. Ele se situa em uma caverna a 100 metros de profundidade perto do prédio principal do CERN, como pode ser visto na Figura~\ref{fig:2T06}, próximo da cidade de Meyrin, na Suíça \cite{cernwebAtlas}.

\begin{figure}[!h]
	\centering
	\includegraphics[width=12cm]{./textuais/atlas/figuras/fig1.png}\\
	\caption{Modelo computacional do Detector ATLAS. Extraído de (www.atlas.ch).}
	\label{fig:2T05}
\end{figure}

\begin{figure}[!h]
	\centering
	\includegraphics[width=10cm]{./textuais/atlas/figuras/fig2.png}\\
	\caption{Vista subterrânea do detector ATLAS . Extraído de (www.atlas.ch).}
	\label{fig:2T06}
\end{figure}

No centro do detector ATLAS, feixes de partículas do LHC colidem gerando produtos da colisão sob a forma de novas partículas, que se espalham em todas as direções. Como este é um detector de uso geral, precisa ser capaz de identificar os mais diversos tipos de partículas. O detector contém seis subsistemas de detecção diferentes dispostos em camadas ao redor do ponto de colisão, no intuito de gravar as trajetórias, momentos e energias das partículas, permitindo, assim, que sua identificação individual seja possível.

Esta seção tem o intuito de prover uma visão geral deste detector, suas característica básicas, detalhes internos e funcionamento, visando um melhor entendimento do ambiente onde está inserida essa dissertação.

\subsection{Sistemas de Coordenadas}\label{subsec:sist_coord}

O detector \ac{ATLAS} tem  formato cilíndrico e, para a identificação da posição das partículas no detector, utiliza-se um sistema de coordenadas pre-estabelecido. Este sistema define o ponto de interação nominal como a origem do sistema de coordenadas; o eixo \emph{z} é definido pela direção do feixe e o plano \emph{xy} é transversal à direção do feixe. Alternativamente, usando coordenadas cilíndricas, o ângulo $\phi$ é medido, como de costume, em torno do eixo do feixe, e o ângulo polar $\theta$ é o ângulo a partir do eixo do feixe \cite{aad2008atlas}. Por fim, a \emph{pseudorapidez} $\eta  =  - \ln \tan \left( {\frac{\theta }{2}} \right)$, sendo $\eta$ e $\phi$ as principais coordenadas do ATLAS, como mostrado na Figura~\ref{fig:2T07}.

\begin{figure}[!h]
	\centering
	\includegraphics[width=12cm]{./textuais/atlas/figuras/sistema_de_coordenadas.pdf}\\
	\caption{Sistema de coordenadas do Detector ATLAS. Extraído de \cite{dos2006sistema}.}
	\label{fig:2T07}
\end{figure}

\subsection{Perfil dos Eventos do ATLAS}\label{subsec:perfil_eve}

O perfil dos eventos de interesse do ATLAS e suas particularidades ditaram as características de construção e modulação do detector, uma vez que a identificação dessas partículas é feita pelas características de sua assinatura, que é uma marca particular, deixada no aparato. Cada componente deste equipamento foi especificado para detectar um conjunto de propriedades das partículas \cite{Lippmann:2011bb}.

Com o conhecimento sobre as peculiaridades de cada parte do detector, podemos entender como é o perfil de alguns desses eventos de interesse, mostrados na Figura~\ref{fig:2T15} e traduzidos abaixo:

\begin{itemize}
  \item Detector de traço: múons, prótons, elétrons, píons e káons;
  \item Calorímetro eletromagnético: múons, elétrons, fótons, prótons, píons e káons;
  \item Calorímetro hadrônico: múons, prótons, nêutrons, píons e káons.
  \item Câmara de múons: Múons;
\end{itemize}

\begin{figure}[h!]
	\centering
	\includegraphics[width=12cm]{./textuais/atlas/figuras/assinatura_particulas_atlas.png}\\
	\caption{Modelo computacional da assinatura das partículas no detector ATLAS. Extraído de (cds.cern.ch).}
	\label{fig:2T15}
\end{figure}

\subsection{Detector Interno}

O \ac{ID} do ATLAS é composto de três partes: uma seção cilíndrica, chamada de Barril(\emph{Barrel}), que cobre a região central $\left( {\left| \eta  \right| \le 1} \right)$, e duas regiões em forma de disco, chamadas de tampas(\emph{Endcaps}), abrangendo as regiões $\left( {1 \le \left| \eta  \right| < 2.5} \right)$.

Os subprodutos das colisões primeiramente cruzam o tubo de feixe, em seguida, as três camadas do Detector de Pixels  (\ac{SPD}), quatro camadas do Detector de Traços baseado em semicondutores (\ac{SCT} e 36 tubos do Detector de Radiação de Transição (\ac{TRT} \cite{barberis2000atlas}.

\begin{itemize}
\item Detector de Pixels  (\ac{SPD}): Esse detector fornece medidas em duas dimensões com alta precisão perto do ponto de interação, que são especialmente importantes para a caracterização de partículas de decaimentos semi-leptônicos;

\item Detector de Traços baseado em semicondutores (\ac{SCT}): Faz 4 pares de medidas por traço e, combinado com o SPD, provê medidas de momento, parâmetro de impacto e posição de vértice;

\item Detector de Radiação de Transição (\ac{TRT}): Esse detector contribui de maneira significativa para a medida precisa do momento de todos os traços, bem como, proporciona uma capacidade inerente de identificação de elétron. \cite{benekos2003atlas}.
\end{itemize}

A Figura~\ref{fig:2T08} mostra a disposição dos detectores e a Figura~\ref{fig:2T09} mostra um corte transversal do ID.

\begin{figure}[h!]
	\centering
	\includegraphics[width=12cm]{./textuais/atlas/figuras/ID1.png}\\
	\caption{Modelo computacional do ID. Extraído de (cds.cern.ch).}
	\label{fig:2T08}
\end{figure}

\begin{figure}[h!]
	\centering
	\includegraphics[width=12cm]{./textuais/atlas/figuras/ID2.png}\\
	\caption{Modelo computacional do ID - corte transversal. Extraído de (cds.cern.ch).}
	\label{fig:2T09}
\end{figure}

\subsection{Calorímetros}

O calorímetro é um dispositivo que absorve toda a energia cinética de uma partícula, que ao colidir com seu material inicia um chuveiro de partículas, cuja interação fornece, ao fim da cadeia, um sinal eletrônico proporcional ao valor da energia depositada \cite{das1994introduction}.

Na interação com o calorímetro cria-se um processo em cascata, onde partículas secundárias são produzidas ao longo do detector. Uma fração dessa energia é entregue na forma de luz de cintilação que produz um sinal detectável.

As características básicas dos calorímetros são:
\begin{itemize}
\item Calorímetros podem ser sensíveis tanto a partículas neutras quanto a carregadas;
\item Pode ser utilizado para identificação de partículas, uma vez que há diferenças na forma de deposição de energia para elétrons, múons e hádrons, por exemplo;
\item Permite tanto medida da energia quanto de trajetória das partículas, devido à sua segmentação;
\item Tempo de resposta rápido (menor que 50 ns), adequando-se a um ambiente com alta taxa de eventos \cite{peralva2012detecccao}.
\end{itemize}

\subsubsection{Chuveiros Eletromagnéticos e Hadrônicos}\label{subsec:chuveiros}

Em física de altas energias podemos destacar dois tipos de chuveiros (ou cascatas) \cite{grupen2008particle}:

\begin{itemize}
\item Chuveiros eletromagnéticos (Figura~\ref{fig:2T10}): são iniciados por elétrons ou fótons com alta energia ao passarem pelo calorímetro. Essas partículas carregadas sofrem interações criando fótons que, por sua vez, se convertem em pares elétron-pósitron. Essa cascata aumenta até a energia dos elétrons ser menor que uma energia crítica;
\item Chuveiros hadrônicos (Figura~\ref{fig:2T11}): decorrem do comprimento de interação nuclear, e geralmente são muito maiores que os chuveiros eletromagnéticos. Seu desenvolvimento lateral é causado pela grande transferência de momento típica de interações nucleares; e são basicamente compostos por píons.
\end{itemize}

\begin{figure}[h!]
	\centering
	\includegraphics[width=13cm]{./textuais/atlas/figuras/photon.png}\\
	\caption{Simulação computacional utilizando algoritmo Corsika do Chuveiro Eletromagnético (100GeV), (a) vista lateral e (b) vista frontal.}
	\label{fig:2T10}
\end{figure}

\begin{figure}[h!]
	\centering
	\includegraphics[width=13cm]{./textuais/atlas/figuras/proton.png}\\
	\caption{Simulação computacional utilizando algoritmo Corsika do Chuveiro Hadrônico (100GeV), (a) vista lateral e (b) vista frontal.}
	\label{fig:2T11}
\end{figure}

\subsubsection{Calorímetro Eletromagnético}\label{subsec:cal_ele}

O \ac{EM} \cite{calorimeter2008construction} é composto de absorvedores de chumbo e eletrodos intercalados em forma de acordeão, sendo utilizado Argônio líquido como material ativo. Esse aparato compõe a parte mais interna no sistema de calorimetria do ATLAS.

\begin{figure}[h!]
	\centering
	\includegraphics[width=12cm]{./textuais/atlas/figuras/liquidargon_segmentation.png}\\
	\caption{Modelo computacional do Calorímetro Eletromagnético. Extraído de \cite{francavilla2012atlas}.}
	\label{fig:2T12}
\end{figure}

A construção desse calorímetro foi dividida em três camadas, sendo a primeira mais segmentada, no intuito de efetuar uma localização precisa da partícula; a segunda é mais profunda e menos segmentada que a primeira; e, por fim, a terceira camada é a menos segmentada e tem a função de absorver completamente a energia da partícula incidente. Essas divisões e segmentações podem ser vistas na Figura~\ref{fig:2T12}.

Como existem perdas de informação devido ao 'material morto' (fios, encapamentos, etc) o calorímetro EM possui um pré-irradiador, que atua na recuperação dessas informações.

\subsubsection{Calorímetro Hadrônico}\label{subsec:cal_had}

O \ac{HAD} do detector ATLAS, chamado de Calorímetro de Telhas (do inglês \emph{Tile Calorimeter}, ou \emph{TileCal}), utiliza placas cintiladoras, em formato de telha, como material ativo e, como material absorvedor, faz o uso de placas de aço com baixo carbono. Esse equipamento é subdivido em 3 partes, como pode ser visto na Figura~\ref{fig:2T13}: o barril \emph{(Tile Barrel)}, situado no região central e dois barris estendidos \emph{(Tile Extended Barrel)}, um em cada lateral do barril. O Tilecal também possui 3 camadas, cada uma segmentada de uma forma diferente \cite{aad2010readiness}.

\begin{figure}[h!]
	\centering
	\includegraphics[width=12cm]{./textuais/atlas/figuras/TileCal.png}\\
	\caption{Modelo computacional do HAD e do EM. Extraído de (cds.cern.ch)}
	\label{fig:2T13}
\end{figure}

Quando partículas 'excitam' as telhas, este material cintilador produz luz, que é transmitida por fibras ópticas até \ac{TFM}. Este instrumento converte a luz em sinal elétrico, que é lido pela eletrônica do TileCal.

\subsection{O Detector de Múons}\label{subsec:cam_muon}

A camada mais externa do detector ATLAS é a câmara de múons. Idealmente, essas partículas são as únicas, detectáveis, capazes de atravessar os calorímetros. O espectrômetro de múons \cite{atlas2010commissioning} circunda o calorímetro e mede as trajetórias dessas partículas, sendo, assim, capaz de medir o seu momento junto com o ID. Esses traços sempre são normais a componente principal do campo eletromagnético, o que torna o resolução do momento transverso rudemente independente de $\eta$.

O sistema de detecção de múons é constituído por milhares de sensores de partículas carregadas, colocados em um campo magnético, produzido por grandes bobinas toroidais supercondutoras. Na Figura~\ref{fig:2T14}, é mostrado o modelo computacional do Detector de Múons.

\begin{figure}[h!]
	\centering
	\includegraphics[width=12cm]{./textuais/atlas/figuras/Camara_Muon.png}\\
	\caption{Modelo computacional da Câmara de Múons do detector ATLAS. Extraído de (cds.cern.ch).}
	\label{fig:2T14}
\end{figure}


\subsection{Sistema de Filtragem do ATLAS}\label{sec:sis_fil}

No intuito de alcançar novas descobertas, utilizando-se de eventos raros, o experimento efetua colisões com uma taxa muito alta de operação, gerando um conjunto muito grande de informações, onde grande parte desses eventos pode ser descartada, a fim de evitar o armazenamento de dados não relevantes ou mesmo já bem explorados \cite{1352047}. Nesse contexto, faz-se necessário um Sistema de Filtragem \emph{Online} que seja capaz de separar os eventos considerados importantes e armazená-los para um análise posterior mediada por algoritmos mais complexos e criteriosos.

Como a Figura~\ref{fig:2T16} apresenta, o sistema de filtragem \emph{online} foi desenvolvido em 3 níveis consecutivos que, juntos, reduzem a taxa de eventos de 40 MHz para 200 Hz \cite{elsing2003configuration}. Esses níveis são o \ac{L1}, \ac{L2} e \ac{EF}, respectivamente.

\begin{itemize}
  \item O primeiro nível de \emph{trigger} é feito em hardware, uma vez que precisa trabalhar com uma latência de $\sim$2$\mu$s. Esse sistema recebe os sinais dos calorímetros e da câmara de múons, separando os conjuntos de eventos que ficaram dentro do limiar de corte estabelecido, chamados de \ac{RoI}, reduzindo a taxa de eventos para $75 kHz$ \cite{gabaldon2012performance};
  \item O segundo nível de filtragem tem sua implementação baseada em softwares operando em uma rede de computadores e sua principal característica é observar as \ac{RoI} pré-definidas pelo L1. Este tem a capacidade de reduzir para $\sim$2 $kHz$ a taxa de eventos;
  \item Já o último nível de filtragem do ATLAS reduz essa taxa para 200 Hz, trabalhando com uma granularidade maior, re-selecionando as informações transmitidas pelo L2.
\end{itemize}


\begin{figure}[h!]
	\centering
	\includegraphics[width=9cm]{./textuais/atlas/figuras/sistema_de_filtragem.pdf}\\
	\caption{Fluxograma do sistema de Trigger Online do ATLAS. Extraído de \cite{dos2006sistema}.}
	\label{fig:2T16}
\end{figure}

%O sistema de filtragem \emph{online} tem o intuito de reduzir a taxa de eventos sem prejudicar a visualização de um possível canal físico relevante, portanto, esta ferramenta deve operar com uma eficiência de detecção alta. Entretanto, isso proporciona um aumento no falso alarme. Consequentemente, a saída deste sistema retorna eventos de interesse juntamente com ruídos de fundo \cite{torres2010sistema}.

Como foi dito na Seção~\ref{sec:ATLAS}, o ATLAS é um detector de uso geral; portanto, os dados armazenados são utilizados em diferentes estudos, que são realizados por filtragem \emph{offline}. Uma vez que este sistema não tem como fator determinante o tempo de processamento, algoritmos mais bem elaborados e específicos para cada tipo de estudo podem ser empregados, possibilitando, assim, uma identificação mais robusta das partículas. O ambiente \emph{offline} é também propício para implementação e testes de novos algoritmos que podem, eventualmente, ser aplicados futuramente no sistema de filtragem \emph{online} do ATLAS.

\chapter{Revisão Bibliográfica}\label{cap:rev}

Nesse capítulo será apresentada uma revisão bibliográfica dos principais conceitos usados para o desenvolvimento desse trabalho. Essa revisão visa mostrar tanto o início das pesquisas nessa área quanto os avanços realizados nos últimos anos. Os tópicos estudados nesse trabalho foram:

\begin{itemize}
	\item Estimação de Densidades não-paramétrica;
	\item Discretização;
	\item Avaliadores de Estimação;
\end{itemize}

\section{Estimação de Densidade não-paramétrica}

Para análise de grandes quantidades de dados e/ou conjuntos multidimensionais de dados é necessário recorrer à ferramentas estatísticas, como estimadores paramétricos, estimadores não-paramétricos ou recursos gráficos \cite{scott2015multivariate}. 

A primeira abordagem citada pode ser considerada a mais eficiente, uma vez que pode-se obter os parâmetros da função de densidade utilizando o conhecimento a priori sobre os dados, entretanto, tal método pode levar a erros de estimação caso a função geradora dos dados seja desconhecida e/ou não pertencer às função já parametrizadas na literatura \cite{sheskin2003handbook}. Já a segunda ferramenta é considerada mais flexível, no sentido de que não é estritamente necessário algum conhecimento anterior sobre os dados a serem analisados, contudo essa abordagem leva a um problema denominado de 'Maldição da Otimização', onde existem parâmetros a serem otimizados que necessitam de um conhecimento prévio da distribuição, nesse momento algumas suposições, baseadas na análise dos dados, podem ser feitas. Essa alternativa de unir um conhecimento prévio ao método de estimação não-paramétrico gera uma discussão, entre alguns autores (e.g.  \cite{marascuilo1977nonparametric}), sobre o uso do termo '\textit{assumption freer}' ao invés do termo 'não-paramétrico'. Por fim, os recursos gráficos se apresentam como uma ferramenta extra que pode mostrar o problema de outros pontos de vista, trazer luz e mostrar o inesperado.

No caso desta tese o desenvolvimento se dá no âmbito da estimação não-paramétrica, uma vez que a intenção desse trabalho é contribuir para a otimização da estimação e classificação de dados quando não se tem conhecimento sobre as funções geradoras.

%\subsection{Estimação de Densidade não-paramétrica}

No campo da estimação de densidade não-paramétrica podemos citar o pesquisador David W. Scott por suas inúmeras contribuições: estudos sobre o histograma \cite{histogramScott}, primeiro estimador não-paramétrico, a concepção do \ac{ASH} \cite{ashScott}, pesquisas relacionadas a otimização de histogramas \cite{scott1979optimal}, outros estimadores não-paramétricos \cite{scott1987biased}, \cite{scott1981monte} e \cite{wang19941}, otimizações do \ac{KDE} \cite{terrell1992variable}, \cite{scott1977kernel} e \cite{scott1985kernel} e estudos sobre estimação de densidades multidimensionais \cite{sain1994cross}, muitos desses tópicos abordados de maneira clara no livro \cite{scott2015multivariate}, que serviu como base para o início das pesquisas dessa tese.

Além disso, analisando os estudos desse tema na literatura temos que os métodos de \ac{KDE} datam de \cite{rosenblatt1956remarks} e \cite{parzen1962estimation}, depois desses podemos citar os trabalhos de  \cite{eubank1988spline}, \cite{hardle1990applied}, \cite{hardle1988smoothing}, \cite{silverman1986density}, \cite{wahba1990spline}, \cite{wand1994kernel} e \cite{jones1996brief} como exemplos de estudos relacionados ao tema de suavização da estimação de densidades, tais métodos são, por muitas vezes, capazes de fornecer uma metodologia útil e eficiente para análise dos dados.

O \ac{KDE} é amplamente utilizado devido a sua simplicidade tanto na implementação quanto na interpretação de seus resultados, em \cite{zambom2012review} o autor apresenta uma revisão sobre esse método, mostrando importantes aspectos teóricos do mesmo. O \ac{KDE} apresenta uso em casos de modelagem, inferência estatística e análise de dados, entretanto, sua utilização pode ser restringida devido ao seu custo computacional, quando comparado a outros métodos \cite{tang2016fast}. 

Com esse detalhe em mente e sabendo que, atualmente, muitos experimentos exigem otimização no sentido de custo computacional e precisão, pode-se citar tópicos como: escolha da largura de banda, sensibilidade a descontinuidades e o próprio custo computacional como escopo de estudo para aprimorar a estimação de densidade baseada no \ac{KDE}. Contudo, ressalta-se que para a otimização de cada tópico citado necessita-se de uma análise profunda e individual, mesmo assim, em muitos casos, pode-se chegar a questões de otimização cíclica. Nos próximos parágrafos serão apresentados alguns dos principais trabalhos desenvolvidos sobre cada tema.


\subsection{Custo computacional}

Mesmo com a publicação de muitos livros e artigos sobre o tema de estimação de densidades nas últimas décadas, as soluções e implementações desenvolvidas com o \ac{KDE} ainda consomem muito tempo. Em casos onde o conjunto de dados a ser analisado não passa de algumas centenas esse problema pode ser desconsiderado, entretanto, quando almeja-se avaliar grandes quantidades de dados \textit{(Big Data)}, essa questão pode ser um obstáculo, principalmente em casos multidimensionais \cite{gramacki2017nonparametric}.

Existem muitos estudos que propõem formas de diminuir o consumo de tempo dos algoritmos de \ac{KDE}, sendo que a maioria faz uso de métodos de aproximação computacional, entretanto, o erro dessas aproximações não pode ser controlado. Com isso em mente os autores de \cite{raykar2010fast} apresentam uma ferramenta baseada em expansão de Taylor, mantendo somente os primeiros termos com o intuito de garantir que o erro causado pelo truncamento da serie permaneça menor que o erro desejado. Em \cite{tang2016fast} é proposto um método de estimação unidimensional e multidimensional baseada em polinômios locais e comparado com o método de \ac{KDE} padrão. Já em \cite{langrene2017fast} é apresentado o método baseado em \textit{Fast Sum Updating} que consiste na ordenação dos dados de entrada e na transição do \textit{kernel} de um ponto de avaliação para outro, atualizando somente os pontos de entrada que não pertenciam a janela selecionada anteriormente. Ainda dentro desse mesmo tema temos \cite{yin2008fast} que aborda esse conteúdo utilizando a teoria de \ac{SBR}. E, de modo geral, o trabalho de \cite{lang2005empirical} apresenta resultados de experimentos testando os métodos \ac{FGT}, \ac{IFGT} e Árvore Dupla (do inglês, \textit{Dual-Tree}) em relação ao tamanho do conjunto de dados, dimensão, erro permitido, além de medidas de tempo de CPU e uso de memória.

\subsection{Escolha da Largura de Banda}

A discussão sobre a seleção de largura de banda (do inglês, \textit{Bandwidth}) vem acontecendo a cerca de três décadas \cite{heidenreich2013bandwidth}. Embora existam outros aspectos a serem analisados na estimativa de densidade, a escolha desse parâmetro não é de forma alguma menos problemática. Isso se torna visível ao ler os muitos estudos empíricos em que os autores apresentam novas propostas normalmente fornecendo simulações em que mostram que o próprio seletor supera alguns dos métodos existentes. Entretanto a seleção da largura de banda é algo muito complexo e nada trivial, nenhum dos muitos métodos de seleção existentes conseguem superar os outros quando se considera as mais variadas densidades a serem estimadas , como mostrado em \cite{heidenreich2013bandwidth} e \cite{turlach1993bandwidth}.

A escolha da largura de banda é sensível a muitos aspectos da distribuição, como \textit{Kurtosis} \cite{decarlo1997meaning}, \textit{Skewness} \cite{mardia1970measures}, caudas exponenciais, descontinuidades, multi-picos, tamanho da amostra, entre outros. Neste contexto, pode-se citar os artigos de \cite{oyang2005data} e \cite{kim2012robust}, o primeiro apresenta a técnica de calcular a largura de banda variável com um modelo mais relaxado, o segundo aborda o tema da ponto de vista da estatística robusta, que tem como principal contribuição a não consideração de pontos fora da densidade (\textit{outliers}) para os cálculo estatísticos e de largura de banda. E as teses de doutorado de \cite{walter1998density}, \cite{duong2004bandwidth} e \cite{schindler2012bandwidth}.

\subsection{Sensibilidade a descontinuidades}


Para a estimação de densidades é comum fazer uso de medidas estatísticas baseadas em momentos. No entanto, quando não é verdadeira a suposição de que a distribuição em questão é uma distribuição Gaussiana, os modelos calculados a partir dessa medida não são ótimos. Um dos casos em que isso pode acontecer é quando os dados contêm pontos distantes ou discrepantes dos parâmetros medidos \textit{outliers}. No sentido estatístico, os \textit{outliers} são amostras de uma população diferente da maioria dos dados. A presença dessas amostras pode ser devido a dois motivos principais: erro experimental ou característica única de alguns objetos. Ambos os tipos de \textit{outliers} são importantes para serem identificados, no entanto, a razão para a sua identificação é diferente em cada caso, no primeiro deseja-se removê-los dos dados para obter resultados corretos da análise, já no segundo, encontrar a explicação para a sua ocorrência no intuito de entender melhor o processo estudado \cite{daszykowski2007robust}.

O tema de identificação de \textit{outliers} e estudo sobre a sensibilidade dos estimadores a esses eventos, recebeu o nome de Estatística Robusta, é vastamente discutido em muitos artigos, como por exemplo: \cite{huber19721972} que faz um revisão dos métodos de estatística robusta; \cite{analytical1989robust} que explora o ponto controverso de que os \textit{outliers} em alguns casos precisam ser considerados e não completamente rejeitados; \cite{rousseeuw2018computation} que aborda o tema de estatística robusta no mundo multivariado; entre outros. Além disso, alguns livros, que podem ser usados como base nesse estudo, foram publicados ao longo dos anos, pode-se citar \cite{rousseeuw2005robust}, \cite{maronna2006robust}, \cite{hampel2011robust} e \cite{olive2018robust}.

É de conhecimento da literatura que um único \textit{outlier} nos dados pode alterar completamente a tendência geral e tornar um modelo inválido para a maioria dos dados. Portanto, faz-se necessário o uso de estimadores robustos em conjunto com abordagens clássicas para melhorar a estimação de densidades e minimizar o erros causados por \textit{outliers}.

\section{Discretização}

A estimativa de densidade via KDE de um conjunto de medidas contínuas, por razões computacionais, é frequentemente representada de forma discreta. Consequentemente, a estimativa direta só acontece para os valores discretos \cite{jones1989discretized} e a interpolação é usada para calcular qualquer outro valor que possa surgir durante as medições. Esse processo insere um erro de estimativa que pode ser minimizado pelo aumento do número de pontos estimados, criando um \textit{trade-off} entre otimização computacional e desempenho de estimativa.

Muitos autores seguem a mesma abordagem de \cite{jones1989discretized}, explorando os diferentes aspectos do processo de discretização e propondo novos métodos para minimizar as adversidades reveladas. Por exemplo, em \cite{fayyad1993multi} é proposto o método Ent-MDLP; em \cite{friedman1996discretizing}, um algoritmo de discretização baseado em \textit{Bayesian Networks} é apresentado; Em \cite{biba2007unsupervised}, os autores propuseram um método não supervisionado de discretização usando o KDE; também utilizando um método não supervisionado, os autores de \cite{schmidberger2005unsupervised} apresentam o estudo da discretização aplicada à estimação da densidade baseada em \textit{Tree-Based}; e em \cite{zhang2007discretization} foi estudado um algoritmo de aprendizado de máquina baseado no critério \textit{Gini}.

Esses trabalhos geralmente se concentram no uso de algoritmos de aprendizado de máquina ou minimização de critérios selecionados para otimizar os vários atributos que existem, como consequência, eles tendem a ter um alto custo computacional ao lidar com grandes quantidades de dados. Além disso, tais estudos abordam o desempenho da discretização através do prisma da classificação e alguns como forma de preprocessamento do conjunto de dados.

O método de discretização mais aplicado é baseado no espaçamento uniforme entre os pontos estimados. Ele trata igualmente todas as regiões de densidade (por exemplo, a cauda da função de densidade é discretizada com a mesma resolução de sua região de maior probabilidade) levando a um erro de estimativa que tende a não ser uniforme em todas as regiões da funções de densidade probabilidade.

\section{Avaliadores de Estimação}

Para avaliar um procedimento de estimação de \ac{PDF}, tornam-se necessárias medidas \textit{goodness-of-fit} que podem ser feitas considerando a distância ou a similaridade entre as funções.

Devido à sua importância, as medições de distância ou similaridade entre duas funções são fundamentais para melhorar a própria estimativa de \ac{PDF} e resolver problemas de classificação, \textit{clustering} e estimativa de densidade espectral. No entanto, existem muitas medidas de distância e similaridade na literatura, como pode ser visto em \cite{deza2006dictionary} e \cite{deza2009encyclopedia}, cada uma mais adequada para um tipo de aplicação do que para o outro.

Em \cite{basseville2013divergence}, foi escrita uma bibliografia anotada sobre medidas de divergência para processamento de dados estatísticos e problemas de inferência. No entanto, mesmo após muitos estudos, há uma demanda contínua para encontrar uma medida ideal de distância/similaridade entre funções para cada tipo de aplicação e suas particularidades.

Motivado pelos estudos apresentados em \cite{cha2007comprehensive}, que aborda o tema das muitas medidas de similaridade calculando a correlação entre as mesmas, os autores de \cite{souza2017study} tiveram como o principal objetivo encontrar uma medida eficiente de distância/similaridade a ser aplicada a métodos não-paramétricos de estimativa de densidade no contexto de classificação, caracterizando as principais medidas de distâncias/similaridade e avaliando suas sensibilidades para regiões da \ac{PDF} separadas.



%\chapter{Identificação de Elétrons}\label{cap:identificacao}

Um bom desempenho na reconstrução e identificação de elétrons é um ingrediente fundamental para o sucesso do programa científico do experimento ATLAS, uma vez que as principais assinaturas dos processos eletrofracos são os léptons \cite{alison2014road} e são utilizados para inúmeras análises, como, por exemplo, as medidas de precisão do modelo padrão, a descoberta do bóson de \emph{Higgs} e a busca por uma nova física além do Modelo Padrão \cite{aad2014electron}.


Os elétrons isolados produzidos em muitos do processos físicos de interesse estão sujeitos a uma grande quantidade de ruídos de fundo provenientes de:
\begin{itemize}
  \item Hádrons identificados equivocadamente;
  \item Fótons convertidos;
  \item Electrons não-isolados originados de decaimentos \emph{heavy-flavour}.
\end{itemize}

Por essa razão, é de primordial importância alcançar uma eficiente identificação de elétrons, sobre todo o detector e, ao mesmo tempo, manter uma grande rejeição de ruído de fundo.

\section{Reconstrução de Elétrons}\label{sec:rec_ele}

O algoritmo que efetua a reconstrução dos elétrons da região central ($|\eta| < 2,5$) do detector ATLAS identifica as energias depositadas no calorímetro EM e as associa aos traços do ID \cite{aad2014electron}, seguindo os três passos abaixo descritos:

\begin{enumerate}
  \item Reconstrução do \emph{Cluster}: o conjunto de células (do inglês \emph{cluster seed}) do EM provém das energias depositadas que contêm um total de energia transversa superior a 2,5 GeV, através de um algoritmo de janela móvel, com janela de tamanho 3x5 em unidades de 0,025 x 0,025 em $\eta$ x $\phi$;
  \item Combinar traço com conjunto de células: um traço e uma célula podem ser ditos combinados se a distância entre o ponto de impacto do traço e o baricentro da célula for $\Delta$$\eta$ < 0,05. E o tamanho de $\Delta$$\phi$ necessariamente precisa estar dentro de uma janela de 0.1;
  \item Candidato a elétron reconstruído: depois de combinar traço-célula, o tamanho do \emph{cluster} é otimizado para $\Delta$$\eta$ x $\Delta$$\phi$ = 3x7 (5x5) barril (tampa). O total da energia do candidato a elétron reconstruído é determinado pela soma de 4 fatores \cite{abat2008expected}:
      \begin{itemize}
        \item A energia estimada depositada na parte frontal do calorímetro EM;
        \item A energia depositada no \emph{cluster};
        \item A energia depositada fora do \emph{cluster} (também chamadas de perdas laterais);
        \item A energia depositada atrás do \emph{cluster} (perdas longitudinais).
      \end{itemize}
\end{enumerate}
	
A eficiência da reconstrução para os elétrons que passam pelo procedimento acima descrito é alta. Nesse estágio, da-se o nome de "\emph{reconstructed electrons }" aos candidatos que foram aprovados nos requisitos de \emph{cluster} e traço.

\subsection{\emph{Trigger} de Elétrons}
	
O sistema de \emph{Trigger} do ATLAS, como já mencionado na Seção~\ref{sec:sis_fil}, é constituído de três níveis, sendo que o L2 e EF juntos compõem a \ac{HLT}. No primeiro nível, selecionam-se somente os elétrons que ultrapassem um limiar de energia e, devido à dependência em $\eta$, esse limiar sofre variações \cite{alison2014road}.

O HLT utiliza as RoI preestabelecidas pelo L1; entretanto, um limiar mais refinado pode ser aplicado, bem como o uso de variáveis discriminantes, que serão apresentadas na Seção~\ref{sec:var_dis}.

Os pontos de operação do \emph{trigger} são definidos em três categorias:

\begin{itemize}
  \item \emph{Trigger} Primário: critérios rígidos são aplicados para coletar eventos de sinal em análise usando elétrons;
  \item \emph{Trigger} de Suporte: coletam amostras de elétrons não polarizados, utilizando basicamente E${_t}$ como critério;
  \item \emph{Trigger} para Monitoramento e Calibração: utilizados para coletar dados no intuito de garantir o correto funcionamento do \emph{trigger} e do detector \cite{aad2012performance}.
\end{itemize}


\section{Variáveis Discriminantes para Identificação de Elétrons}\label{sec:var_dis}

Tanto nas análises \emph{online} quanto \emph{offline}, critérios adicionais são aplicados no intuito de garantir uma melhor pureza dos elétrons reconstruídos. Estes critérios são informações retiradas dos calorímetros e do detector interno \cite{atlas2011expected} a partir de um conjunto de variáveis discriminantes, que podem ser dividas em:

\begin{itemize}
  \item Variáveis de Calorimetria;
  \item Variáveis de Traço;
  \item Variáveis de Traço-Calorimetria;
  \item Variáveis de Isolamento.
\end{itemize}

\subsection{Variáveis de Calorimetria}

As variáveis de calorimetria utilizam a fina segmentação lateral e longitudinal dos calorímetros do detector ATLAS:

\begin{itemize}
  \item Variável de vazamento hadrônico, R${_{had_1}}$:

  Definida como a razão entre as energias transversas da primeira camada do calorímetro hadrônico e do \emph{cluster}. Elétrons reais depositam mais energia no EM do que no HAD, apresentando assim valores pequenos de R${_{had_1}}$;

  \item Variável de largura em $\eta$ na segunda camada, ${W_{\eta 2}}$:

  É a medida da largura do chuveiro em $\eta$ ponderada pelo \ac{RMS} da distribuição em $\eta$ na segunda camada do EM. Esta variável contribui para suprimir ruído de fundo de jatos e conversões de fótons, que tendem a ter chuveiros maiores do que elétrons verdadeiros;

  \item Variável de largura do chuveiro, R${_\eta}$:

  É definida como a razão entre energia de uma janela 3x7 sobre uma janela 7x7, na segunda camada de amostragem.  Ruídos de fundo tendem a ter uma maior fração de energia fora do núcleo 3x7, resultando em baixos valores de R${_\eta}$.

  \item Variável de largura do chuveiro, R${_\phi}$:

  Semelhante à variável R${_\eta}$; entretanto, definida como a razão entre a energia em uma janela 3x3 sobre uma janela 3x7.

  \item Variável de largura do chuveiro na primeira camada do calorímetro, w${_{stot}}$:

  Mede a largura do chuveiro, que ajuda na identificação de elétrons porque apresenta maiores valores para ruído de fundo.

  \item Variável de razão de energia, E${_{ratio}}$.

  Também é utilizada para diminuir o ruído de fundo. É definida utilizando as células correspondentes às duas maiores energia nas camadas. Ruídos de fundo tendem a ter múltiplas incidências de partículas associadas, apresentando assim valores de E${_{ratio}}$ menores do que partículas de sinal.

  \item Fração de energia da terceira camada do EM, f${_3}$:

  Essa variável tende a ser menor para elétrons do que para ruído de fundo, uma vez que os elétrons não penetram tão profundamente no calorímetro.

  \item Fração de energia nas camadas do EM, f${_1}$:

  Essa variável é definida como a razão de energia depositada nas camadas sobre a energia total do EM.

\end{itemize}

A Figura~\ref{fig:3T01} mostra algumas das distribuições das variáveis acima apresentadas, bem como os vários tipos de ruídos de fundo.

\begin{figure}[h!]
	\centering
	\includegraphics[width=14cm]{./textuais/identificacao/figuras/fig_1_variaveis_eletron.pdf}\\
	\caption{Variáveis de identificação de elétrons no calorímetro, formato do chuveiro, apresentados separadamente para sinal e os vários tipos de ruídos de fundo. As variáveis apresentadas são: (a) vazamento hadrônico R${_{had}}$, (b) de largura em eta no segundo W${_2}$ amostragem, (c) R${_\eta}$, (d) largura em $\eta$ nas w${_{s,tot}}$, pequeno, e (e) E${_{ratio}}$. Extraído de \cite{alison2014road}.}
	\label{fig:3T01}
\end{figure}

Essas variáveis são dependentes de $\eta$ e E${_t}$. Em $\eta$ devida à geometria dos calorímetros, que apresentam alguns pontos com menor resolução, como, por exemplo, a região, definida como \emph{crack}, que se encontra entre o barril e a tampa, 1,37 < |$\eta$| < 1,52, e que muitas vezes, é excluída de análises por conta da baixa resolução. Por outro lado, o poder de discriminação dessas variáveis melhora com o aumento de E${_t}$, uma vez que a largura do chuveiro tende a diminuir e o ruído de fundo tende a ter uma menor dependência de E${_t}$.

\subsection{Variáveis de Traço} \label{sec:variaveis_traço}

As variáveis de traço são provenientes do ID e podem ser utilizadas de forma a complementar as do calorímetro.
	
\begin{itemize}
  \item Número de \emph{hits} no detector Pixel (nPixHits) e Número combinado de \emph{hits} do Pixel e detectores de SCT (nSixHits):

      As camadas de detectores que são atravessadas por fótons antes de serem convertidos não têm traços associados a eles. Isso resulta em um menor número de \emph{hits} no detector de Pixels e SCT, do que os elétrons verdadeiros.

  \item Número de \emph{hits} na primeira camada do Detector de Pixels ou \emph{B-layer}:

  Essa variável apresenta uma sensibilidade a todas as conversões que ocorrem depois da primeira camada do Detector de Pixels, sendo bastante efetiva na redução de ruído de fundo.

  \item Parâmetro de impacto transverso, D${_0}$:

  Mede a distância mais próxima do traço do elétron até o vértice primário e possibilita a separação de conversões, dado que estes podem ter traços deslocados significativamente dos pontos de interação.

  \item Significância do parâmetro de impacto transverso, ${\sigma _{{d_0}}}$:

  Mede a relevância da distância mais próxima do traço do elétron até o vértice primário.

  \item \emph{Flag} de conversão, ou "bit conversão":

  É definido se o traço do elétron corresponde a um vértice de conversão. Reduz o número de elétrons reconstruídos de conversões, entretanto não é tão eficiente para elétrons verdadeiros.

  \item Fração de \emph{hits} de alto \emph{threshold} no TRT:

  Essa variável é uma das mais poderosas contra ruído de fundo provenientes de hádrons, em razão de mostrar a fração das detecção que passaram o limiar do detector TRT, indicando a presença de radiação de transição de fótons.

\end{itemize}

Na Figura~\ref{fig:3T02} são apresentadas algumas das variáveis de traço. Diferente das variáveis de calorimetria, as variáveis de traço são independentes de $\eta$ e E${_t}$, com exceção do TRT, e são pouco dependentes do \emph{pileup}.

\begin{figure}[h!]
	\centering
	\includegraphics[width=14cm]{./textuais/identificacao/figuras/fig_2_variaveis_eletron.pdf}\\
	\caption{Variáveis de identificação elétron no ID, agrupados em sinal e vários tipos de ruídos de fundo. As variáveis apresentadas são: (a) número de \emph{hits} no detector Pixel, (b) número combinado de \emph{hits} do Pixel e detectores de SCT, (c) parâmetro de impacto transverso D${_0}$, (d) \emph{flag} de conversão, ou "bit conversão", e (e) fração \emph{hits} de alto \emph{threshold} no TRT. Extraído de \cite{alison2014road}.}
	\label{fig:3T02}
\end{figure}

\subsection{Variáveis de combinação Traço-Calorimetria}

Ao combinar as informações de traço e calorimetria, tem-se variáveis adicionais para discriminação de ruído de fundo. Essas são apresentadas na Figura~\ref{fig:3T03}.

\begin{figure}[h!]
	\centering
	\includegraphics[width=12cm]{./textuais/identificacao/figuras/fig_3_variaveis_eletron.pdf}\\
	\caption{Variáveis combinadas de traço-calorimetria, mostrando a separação de vários tipos de background. As variáveis mostradas são: (a) diferença entre o traço e o \emph{cluster} de energia em $\eta$, (b) diferença entre o traço e o \emph{cluster} de energia em $\phi$, e (c) razão da energia medida no calorimetro com o momento medido no traço. Extraído de \cite{alison2014road}.}
	\label{fig:3T03}
\end{figure}

\begin{itemize}
  \item Variável de diferença entre o traço e o \emph{cluster} de energia em $\eta$, $\Delta$$\eta_1$:

  A comparação é feita extrapolando o traço até o calorímetro EM e esta distribuição é mais reduzida para os elétrons reais, portanto, a exigência de valores pequenos de $\Delta$$\eta$ reduz o ruído de fundo.

  \item Variável de diferença entre o traço e o \emph{cluster} de energia em $\phi$, $\Delta$$\phi_2$:

  Semelhante a variável descrita anterior, entretanto, menos discriminante devido aos fótons da radiação de Bremsstrahlung causarem uma diferença entre a posição do traço e o cluster em $\phi$.

  \item Variável de diferença entre o traço e o \emph{cluster} de energia em $\phi$, reescalada, $\Delta$$\phi_{res}$:

  É a variável $\Delta$$\phi_2$, porém com o momento de traço reescalado para a energia do \emph{cluster} depois da extrapolação para a camada central.

  \item Variável de razão entre a energia medida no calorímetro pelo momento determinado no ID, E/P:

  Como os hádrons não depositarão toda sua energia no EM, uma fração será depositada no HAD. A exigência de que E/P seja consistente com a expectativa de um elétron real pode suprimir tanto hádrons e quanto conversões.

\end{itemize}

\subsection{Variáveis de Isolamento}

Por último, as variáveis de isolamento também são utilizadas para discriminar sinal e ruído de fundo, são elas:

\begin{itemize}
  \item Et$_{cone}$;
  \item Pt$_{cone}$.
\end{itemize}

O isolamento é medido pela quantidade de energia próxima do elétron reconstruído, uma vez que elétrons de ruído de fundo são produzidos juntamente com outras partículas, o que os leva a ter maiores valores nessas variáveis. Dessa forma, essas variáveis conseguem ajudar na identificação de sinal e ruído de fundo

\section{Algoritmos Offline de Referência para a Identificação de Elétrons}

A colaboração ATLAS utiliza alguns algoritmos \emph{offline} para identificação de elétrons. Nessa seção, serão apresentados dois algoritmos, o e/$\gamma$, que é o padrão da colaboração e o algoritmo baseado em Verossimilhança, (do inglês, \emph{Likelihood}), que é a metodologia utilizada nessa dissertação.

\subsection{ATLAS e/$\gamma$}

O algoritmo e/$\gamma$ de seleção e identificação de elétrons, tem como princípio o corte baseado nas variáveis discriminantes. A utilização deste método como padrão traz a vantagem de compartilhamento e cruzamento de análises entre as diversas linhas de pesquisa no ATLAS \cite{aad2012performance}.

Como o intuito desta ferramenta é de ser compatível com o maior número possível de pesquisas físicas, três pontos de operação são disponibilizados \cite{alison2014road}:

\begin{description}
\item[\emph{Loose}] Nível de detecção de sinal elevado; entretanto, com a pior rejeição de ruído de fundo entre os três;
\item[\emph{Tight}] Melhor rejeição de ruído de fundo, por conseguinte, menor nível de eficiência entre os três.
\item[\emph{Medium}] Apresenta o ponto de equilíbrio entre os dois primeiros, com nível de rejeição de ruído melhor que o \emph{Loose} e eficiência melhor do que o \emph{Tight}.
\end{description}

Esses pontos de operação são configurados de forma que o \emph{Loose} seja um subconjunto de \emph{Medium} que é um subconjunto do \emph{Tight}, como mostra a Tabela~\ref{tab:01}; entretanto, com valores de cortes um pouco diferentes.

\begin{table}
  \centering
  \caption{Sumário das variáveis usadas nos critérios \emph{Loose++}, \emph{Medium++} e \emph{Tight++} do isEM++. Extraído de \cite{alison2014road} }\label{tab:01}
\begin{tabular}{c}

\textbf{Loose++}	\\ 	\hline
Shower shapes: R${_\eta}$, R${_{had1}}$ (R${_{had}}$, w${_2}$, E${_{ratio}}$,  w${_{s,tot}}$	\\ 	
Track quality	\\ 	
|${\Delta\eta}$| < 0.015	\\ 	
	\\ 	
\textbf{Medium++}	\\ 	\hline
Shower shapes: Same variables as Loose++, but at tighter values	\\ 	
Track quality	\\ 	
|${\Delta\eta}$| < 0.005	\\ 	
N${_{BL}}$ $\ge$ 1 for ${\eta}$ < 2.01	\\ 	
N${_{Pix}}$ > 1 for ${\eta}$ > 2.01	\\ 	
Loose TRT HT fraction cuts	\\ 	
|d0| < 5 mm	\\ 	
	\\ 	
\textbf{Tight++}	\\ 	\hline
Shower shapes: Same variables as Medium++, but at tighter values	\\ 	
Track quality	\\ 	
|${\Delta\eta}$|  < 0.005	\\ 	
N${_{BL}}$ $\ge$ 1 for all ${\eta}$	\\ 	
N${_{Pix}}$ > 1 for ${\eta}$ > 2.01	\\ 	
Tighter TRT HT fraction cuts	\\ 	
|d0| < 1 mm	\\ 	
E/P requirement	\\ 	
|${\Delta\phi}$|   requirement	\\ 	
Conversion bit	\\ 	

	
  %\hline
\end{tabular}
\end{table}
	
O perfil dos pontos de operação do e/$\gamma$, em 2011,  ganhou uma versão mais atualizada, e seus pontos de operação são chamados: \emph{Loose++}, \emph{Medium++} e \emph{Tight++}.


\subsection{Verossimilhança} \label{sec:likelihood}

Dentre as técnicas multivariadas existentes, a verossimilhança apresenta a vantagem de uma construção simples, no caso de independência entre as variáveis.

O método de verossimilhança faz uso de \ac{PDF} das variáveis discriminantes de sinais e de ruído de fundo para encontrar a probabilidade total. No algoritmo de verossimilhança do ATLAS, as PDF foram feitas por uma ferramenta da colaboração chamada \emph{\ac{TMVA} adaptative KDE}, que utiliza um método não-paramétrico chamado \ac{KDE} \cite{therhaag2012tmva}.


\subsubsection{O método da Verossimilhança}

O método de verossimilhança é uma função dos parâmetros de um modelo estatístico que permite inferir sobre o seu valor a partir de um conjunto de observações. No caso de identificação de elétrons do ATLAS, esse conjunto de observações são as variáveis discriminantes. Primeiramente, são construídas as PDF a partir dos dados de sinal e ruído de fundo. Com a consideração de independência entre as variáveis, as probabilidades conjuntas de sinal e ruído de fundo podem ser calculadas através de uma simplificação do método de verossimilhança, como mostram as equações~\ref{eq:04} e \ref{eq:05} \cite{atlas2014electron}.

\begin{equation}\label{eq:04}
    {L_s}\left( x \right) = \prod\limits_{a = 1}^m {{P_{s,a}}({x_a})}
\end{equation}

\begin{equation}\label{eq:05}
    {L_b}\left( x \right) = \prod\limits_{a = 1}^m {{P_{b,a}}({x_a})}
\end{equation}
onde ${{P_{s,a}}({x_a})}$ e ${{P_{b,a}}({x_a})}$ são as probabilidades associadas a cada uma das $m$ variáveis ($x_i$) do evento analisado. L$_{s}$ e L$_{b}$ são os valores da multiplicação da verossimilhança para sinal e ruído de fundo.

Com as duas probabilidades conjuntas calculadas, faz-se o discriminante, utilizando a equação \ref{eq:06}, onde dL é o discriminante.

\begin{equation}\label{eq:06}
    dL = \frac{{Ls}}{{Ls + Lb}}
\end{equation}

Construir o menu para a verossimilhança consiste em: escolher as variáveis, selecionar cortes adicionais e definir o valor de limiar do discriminante \cite{atlasdescription}, sendo que a eficiência da verossimilhança será o resultado da eficiência do discriminante e os cortes adicionais combinados.

Uma das principais diferenças entre o algoritmo e/$\gamma$ e a verossimilhança está nos eventos de cauda da PDF, uma vez que o primeiro efetua cortes rígidos, impossibilitando assim a classificação destes \cite{atlasdescription}.

\subsubsection{Verossimilhança para Elétrons no ATLAS}
	

O método de verossimilhança para identificação de elétrons apresenta cinco pontos de operação: \emph{Very Tight, Tight, Medium, Loose, Very Loose}, cada um com diferentes níveis de rejeição de ruído e eficiência de sinal, e estes pontos diferem entre si pelas variáveis e limiar dos cortes adicionais, como mostra a Tabela~\ref{tab:06}.

\begin{table}[h!]
  \centering
  \caption{Variáveis usadas na construção da verossimilhança para diferentes pontos de operação. Extraído de \cite{atlasdescription}.}\label{tab:06}
\begin{tabular}{c|c|c|c}

Menu	&	VERY TIGHT, TIGHT	&	MEDIUM	&	LOOSE, (VERY LOOSE)	\\ 	\hline
Variáveis da	&	R${_{Had}}$	&	R${_{Had}}$	&	R${_{Had}}$	\\ 	
Verossimilhança	&	R${_{\eta}}$	&	R${_{\eta}}$	&	R${_{\eta}}$	\\ 	
	&	F${_{HT}}$	&	F${_{HT}}$	&	F${_{HT}}$	\\ 	
	&	${\Delta\eta_{1}}$	&	${\Delta\eta_{1}}$	&	${\Delta\eta_{1}}$	\\ 	
	&	W${_{\eta2}}$	&	W${_{\eta2}}$	&	W${_{\eta2}}$	\\ 	
	&	f${_{1}}$	&	f${_{1}}$	&	f${_{1}}$	\\ 	
	&	f${_{3}}$	&	f${_{3}}$	&	f${_{3}}$	\\ 	
	&	E${_{ratio}}$	&	E${_{ratio}}$	&	E${_{ratio}}$	\\ 	
	&	R${_{\phi}}$	&	R${_{\phi}}$	&	R${_{\phi}}$	\\ 	
	&	$\Delta p/p$	&	$\Delta p/p$	&	$\Delta p/p$	\\ 	
	&	${\Delta\phi_{Res}}$	&	${\Delta\phi_{Res}}$	&	${\Delta\phi_{Res}}$	\\ 	
	&	d${_{0}}$	&	d${_{0}}$	&		\\ 	
	&	${\sigma _{{d_0}}}$	&	${\sigma _{{d_0}}}$	&		\\ 	\hline
Cortes	&	nSiHits $\ge$  7	&	nSiHits $\ge$  7	&	nSiHits $\ge$  7	\\ 	
Adicionais	&	nPixHits $\ge$ 2	&	nPixHits $\ge$ 2	&	nPixHits $\ge$ 2 ($\ge$ 1)	\\ 	
	&	Blayer	&	Blayer	&	Blayer (no Blayer)	\\ 	
	&	!(isConv)	&		&		\\ 	\hline
Comparar com	&	isTightPlusPlus	&	MediumPlusPlus	&	isLoosePlusPlus	\\ 	
	&		&		&	Multilepton	\\ 	

\end{tabular}
\end{table}

%\chapter{DESENVOLVIMENTO} \label{cap:desenvolvimento}
\vspace{-2cm}
Neste capítulo serão descritos alguns detalhes da construção dos métodos e do algoritmo para avaliação de sua performance, além disso, serão mostradas as dificuldades encontradas na aplicação prática desses métodos. A discretização da estimação de densidades de variáveis discriminantes pode influenciar de forma direta na tarefa de classificação, entretanto, este trabalho se concentrou somente no impacto desses métodos na estimação, entendendo que um menor erro de estimação pode levar a uma melhor classificação.

\section{Conjunto de dados}

Um dos objetivos desse trabalho é otimizar o desempenho dos algoritmos de estimação de densidade via KDE que fazem uso de métodos de discretização e posteriormente essas estimações serão usadas para a identificação e classificação de eventos. Portanto o conjunto de dados aqui escolhido tem por base as estimações que podem ser encontradas em alguns dos experimentos de física de partículas mais importantes atualmente, como o ATLAS e CMS.

Nas Figuras \ref{fig:15} e \ref{fig:16} são mostrados casos representativos de variáveis usadas para a identificação de elétrons nos experimentos CMS e ATLAS, respectivamente. Para o CMS, pode-se ver o perfil dos elétrons e do ruído de fundo, bem como a diferença entre dados reais e simulados. Já na Figura \ref{fig:16}, além do perfil dos elétrons, é mostrado também as várias formas de ruído de fundo para elétrons no ATLAS.


\begin{figure}[H]
	\begin{center}
		\includegraphics[width=0.8\linewidth]{./figuras/variaveisCMS.png}
		\caption{Caso representativo, variáveis de identificação de elétrons, do experimento CMS: variável $\Delta\eta$, forma do chuveiro $\sigma_{\eta\eta}$ e distribuição de energia-momento $1/E_{SC}-1/p$. Extraído de \cite{cms2015performance}.}\label{fig:15} 
	\end{center}
\end{figure}

Pode-se considerar que a identificação de elétrons em experimentos de física de altas energias é de certa forma similar, respeitando as particularidades de cada detector. Portanto, é de se esperar que a otimização de algoritmos de identificação de elétrons estudada em um conjunto de dados possa ser reproduzida, considerando as especificidades de cada experimento, em um outro conjunto de dados, fazendo com que o estudo para melhoria do desempenho desse processo seja de suma importância.

\begin{figure}[H]
	\begin{center}
		\includegraphics[width=0.8\linewidth]{./figuras/variaveisATLAS.pdf}
		\caption{Caso representativo, variáveis de identificação de elétrons, do experimento ATLAS, no calorímetro, formato do chuveiro, apresentados separadamente para sinal e os vários tipos de ruídos de fundo. As variáveis apresentadas são: (a) vazamento hadrônico R${_{had}}$, (b) de largura em eta no segundo W${_2}$ amostragem, (c) R${_\eta}$, (d) largura em $\eta$ nas w${_{s,tot}}$, pequeno, e (e) E${_{ratio}}$. Extraído de \cite{alison2014road}.}\label{fig:16}
	\end{center}
\end{figure}

Pode-se observar que as variáveis discriminantes destes experimentos apresentam distribuições que muitas vezes se assemelham a algumas distribuições conhecidas na literatura, como a distribuição \textit{Gaussiana} e a \textit{Lognormal}. Sendo assim, com o intuito de realizar as análises em um ambiente controlado foram escolhidas as distribuições análiticas \textit{Gaussiana} e \textit{Lognormal}, sendo que a última parametrizada de 6 maneiras diferentes e três \textit{datasets} discretos diferentes cuja \textit{binagem} foi feita usando o método \ac{FD}.

A distribuição normal foi definida com média $ \mu =0 $ e desvio padrão $ \sigma = 1 $, descrita pela Equação~\eqref{equ:Normal} e ilustrada na Figura~\ref{fig:Gaussiana}, devido ao fato de tais parâmetros não interferirem na forma desta distribuição.


Já a distribuição \textit{Lognormal} com média $ \mu = 0 $ e desvio padrão $\sigma$ com valores $ 0.01 $, $ 0.25 $, $ 0.5 $, $ 1 $, $ 1.25 $ e $ 1.5 $, descrita pela Equação~\eqref{eq:Lognormal} e ilustrada na Figura~\ref{fig:Lognormal}, representada por L($ \mu $,$ \sigma $).

\begin{equation}
{\displaystyle f_{L_X}(x;\mu ,\sigma )={\frac {1}{x\sigma {\sqrt {2\pi }}}}\cdot e^ {-\frac {\left(\ln(x)-\mu \right)^{2}}{2\sigma ^{2}}}}
\label{eq:Lognormal}
\end{equation}

%e com três \textit{Datasets} discretos diferentes, o primeiro sendo de uma distribuição normal com média nula e desvio padrão unitário representado na Figura~\ref{fig:randn}, o segundo será também uma distribuição normal com os mesmos parâmetros da primeira mas com a diferença que este irá possuir \textit{outliers}, ou seja, alguns pontos distantes da região de interesse ilustrado pela Figura~\ref{fig:randn_out} e, por fim, a ultima será uma distribuição Lognormal com média nula e desvio padrão de 0.5, ilustrado pela Figura~\ref{fig:randlog}.


%Neste capítulo, será apresentado as propostas de métodos de discretização, sua descrição e equacionamento. Além disso, o contexto em que estes métodos serão avaliados também será mostrado.

%Como o objetivo deste trabalho é validar apenas os efeitos da discretização no contexto da estimação de \ac{PDF}, o erro de estimação causado por este processo é medido entre a saída do processo de discretização em si e a função usada para gerar os dados. As funções aqui testadas serão baseadas em distribuições Gaussianas ou Lognormais com diferentes variâncias, bem como suas funções analíticas ou através de dados gerados.


%Estes cinco métodos serão mostrados utilizando \ac{PDF}s analíticas Gaussiana, com média $\mu = 0$ e desvio padrão $\sigma = 1$, descrita pela Equação~\eqref{equ:Normal} e ilustrada na Figura~\ref{fig:Gaussiana}, Lognormal, com $\mu = 0$  Todos os \textit{data sets} possuem mil eventos e foram gerados utilizando a biblioteca \textit{numpy} do \textit{software} \textit{Python} e com o número de \textit{binagem} definido pelo \ac{FD} que é um estimador robusto que leva em conta a variabilidade dos dados e o tamanho dos mesmos. Todos os métodos testados terão o número de estimação $N = 25$ para uma melhor visualização.

%Estes cinco métodos serão demostrados utilizando-se a \ac{PDF} Gaussiana com média $\mu = 0$ e desvio padrão $\sigma = 1$, descrita pela Equação~\eqref{equ:Normal} e ilustrada na Figura~\ref{fig:Gaussiana},  e a \ac{PDF} Lognormal com $\mu = 0$ e desvio padrão $\sigma$ com valores $0.01, 0.25, 0.5, 1, 1.25$ e $1.5$, descrita pela Equação~\eqref{eq:Lognormal} e ilustrada na Figura~\ref{fig:Lognormal}, além disso, estes métodos também s ambas com o numero de pontos $N = 25$.


\begin{figure}[H]
	\centering
	\begin{subfigure}[b]{0.3\textwidth}
		\centering 
		\includegraphics[width=\textwidth]{./figuras/log_sigma_001.png}
		\caption{}
		\label{fig:sig001}
	\end{subfigure}
	\hfill
	\begin{subfigure}[b]{0.3\textwidth}
		\centering 
		\includegraphics[width=\textwidth]{./figuras/log_sigma_025}
		\caption{}
		\label{fig:sig025}
	\end{subfigure}
	\hfill
	\begin{subfigure}[b]{0.3\textwidth}
		\centering 
		\includegraphics[width=\textwidth]{./figuras/log_sigma_05}
		\caption{}
		\label{fig:sig050}
	\end{subfigure}
	
	\begin{subfigure}[b]{0.3\textwidth}
		\centering 
		\includegraphics[width=\textwidth]{./figuras/log_sigma_1}
		\caption{}
		\label{fig:sig100}
	\end{subfigure}
	\hfill
	\begin{subfigure}[b]{0.3\textwidth}
		\centering 
		\includegraphics[width=\textwidth]{./figuras/log_sigma_125}
		\caption{}
		\label{fig:sig125}
	\end{subfigure}
	\hfill
	\begin{subfigure}[b]{0.3\textwidth}
		\centering 
		\includegraphics[width=\textwidth]{./figuras/log_sigma_15}
		\caption{}
		\label{fig:sig150}
	\end{subfigure}
	
	\caption{Ilustração das curvas Lognormais construidas com diferentes parâmetros: (a) L(0,0.01); (b) L(0,0.25); (c) L(0,0.25); (d) L(0,1); (e) L(0,1.25); e (f) L(0,1.5)}
	\label{fig:Lognormal}
\end{figure}

Os \textit{datasets} escolhidos possuem mil eventos, média $ \mu = 0 $ e são baseados na distribuição Gaussiana com desvio padrão $ \sigma = 1 $ representado por randN($ \mu,\sigma $) sem \textit{outliers} ilustrado na Figura~\ref{fig:randn} e com \textit{outliers} em $ \pm 25 $ ilustrado  na Figura~\ref{fig:randn_out} e, por fim, baseado na distribuição Lognormal com desvio padrão $ \sigma = 1 $ representado por randL($ \mu,\sigma $) ilustrado na Figura~\ref{fig:randlog}.

\begin{figure}[H]
	\centering
	\begin{subfigure}[b]{0.27\textwidth}
		\centering 
		\includegraphics[width=\linewidth]{./figuras/datanormal_0}
		\caption{}
		\label{fig:randn}
	\end{subfigure}
	\hfill
	\begin{subfigure}[b]{0.27\textwidth}
		\centering 
		\includegraphics[width=\linewidth]{./figuras/datanormal_25}
		\caption{}
		\label{fig:randn_out}
	\end{subfigure}
	\hfill
	\begin{subfigure}[b]{0.27\textwidth}
		\centering 
		\includegraphics[width=\linewidth]{./figuras/datalognormal_0}
		\caption{}
		\label{fig:randlog}
	\end{subfigure}
	
	\caption{Histograma dos dados gerados sendo eles: (a) Gaussiana com $\mu = 0$ e $\sigma$ = 1; (b) Gaussiana com $\mu = 0$, $\sigma = 1$ e \textit{outlier} em $\pm 25$; (c) Lognormal com $\mu = 0$ e $\sigma = 1$.}
	\label{fig:data}
\end{figure}





\section{Algoritmo}

O algoritmo para comparação e validação dos métodos de discretização de estimação construído nesse trabalho pode ser resumido pelo diagrama de blocos da Figura \ref{fig:08}, sendo testado de duas maneiras diversas: utilizando somente a função analítica e usando os dados gerados a partir das funções geradoras.

\begin{figure}[H]
	\begin{center}
     	\includegraphics[width=0.6\linewidth]{./figuras/algoritmo1}
		\caption{Diagrama de blocos do algoritmo para validação dos métodos de discretização.}\label{fig:08}
	\end{center}
\end{figure}


\begin{description}
	\item[Função Analítica] Inicialmente uma função geradora é escolhida, sua discretização é feita utilizando os métodos apresentados. Essa discretização é utilizada para calcular todos os pontos da curva original (utilizando interpolação) e por fim faz-se o cálculo da distância entre as duas curvas (analítica e discretizada) utilizando a métrica de distancia chamada L1, que é um caso particular da Equação \eqref{eq:4T06}.
	
	\begin{equation}\label{eq:4T06}
	{L_p} = {\left( {\int {{{\left| {f(x) - g(x)} \right|}^p}}\cdot dx } \right)^{{\raise0.7ex\hbox{$1$} \!\mathord{\left/
					{\vphantom {1 p}}\right.\kern-\nulldelimiterspace}
				\!\lower0.7ex\hbox{$p$}}}}
	\end{equation}
	
	Onde $p$ é o parâmetro a ser escolhido. No caso mais simples, $p=1$, a equação~\eqref{eq:4T06} se torna a equação~\eqref{eq:4T15}.
	
	\begin{equation}\label{eq:4T15}
	{L_1} = {\int {\left| {f(x) - g(x)} \right|} \cdot dx}
	\end{equation}
	
	A equação~\eqref{eq:4T15} é chamado de \ac{IAE} ou distância L1.
	
	\item[Dados gerados] Nesse caso, a única diferença é que o cálculo da discretização é feito usando como base a distribuição de eventos aleatórios geradas pela função geradora escolhida.
\end{description}

%\color{red} PAREI AQUI \color{black}

\section{Demostração dos métodos de discretização}

Com o intuito de validar os algoritmo e os métodos de discretização serão apresentados nessa seção o funcionamento desses métodos para a função gaussiana, função \textit{lognormal} e com dados gerados computacionalmente. E, com o objetivo de demostrar de forma mais clara o efeito desses métodos e sua dependência ao número de pontos de estimação escolhidos, os métodos serão avaliados variando o número de estimação entre $ N = 15 $ e $ N = 25 $.

\subsection{Método \textit{Linspace}}

De acordo com o princípio de funcionamento do método de discretização \textit{Linspace} espera-se que este entregue resultados satisfatórios em distribuições com variações mais lentas, como mostrado nas Figuras~\ref{fig:lin_norm15}, \ref{fig:lin_norm25}, \ref{fig:lin_norm15_data} e \ref{fig:lin_norm25_data}. Já para conjunto de dados que apresenta variações mais rápidas ou \textit{outliers}, como mostrado nas Figuras~\ref{fig:lin_log15}, \ref{fig:lin_log25}, \ref{fig:lin_norm15_data_out} e \ref{fig:lin_norm25_data_out}, este método não alcança uma boa representação da PDF.


\begin{figure}[H]
	\centering
	\begin{subfigure}[b]{0.45\textwidth}
		\centering 
		\includegraphics[width=\linewidth]{./figuras/Linspace_normal_15}
		\caption{}
		\label{fig:lin_norm15}
	\end{subfigure}
	\hfill
	\begin{subfigure}[b]{0.45\textwidth}
		\centering 
		\includegraphics[width=\linewidth]{./figuras/Linspace_normal_25}
		\caption{}
		\label{fig:lin_norm25}
	\end{subfigure}
	\\
	\begin{subfigure}[b]{0.45\textwidth}
		\centering 
		\includegraphics[width=\linewidth]{./figuras/Linspace_lognormal_15}
		\caption{}
		\label{fig:lin_log15}
	\end{subfigure}
	\hfill
	\begin{subfigure}[b]{0.45\textwidth}
		\centering 
		\includegraphics[width=\linewidth]{./figuras/Linspace_lognormal_25}
		\caption{}
		\label{fig:lin_log25}
	\end{subfigure}
	
	\caption{Discretização utilizando o método de \textit{Linspace}: (a) N(0,1) com $N = 15$, (b) N(0,1) com $N = 25$, (c) L(0,1) com $N = 15$ e (d) L(0,1) com $N = 25$.}
	\label{fig:normlin}
\end{figure}




Entretanto, pode-se perceber que com o aumento do número de pontos de estimação (de $N = 15$ para $N = 25$) o desempenho deste método apresenta uma melhora significativa. Portanto, fica claro que é possível alcançar uma boa performance com o método de \textit{Linspace}. Mas, há de se comentar que o aumento do número de pontos de estimação é um fator decisivo no custo computacional desses algoritmos, e, além disso, aumenta a quantidade de informação a ser armazenada ou transmitida. Apesar da distribuição randL(0,1) ser baseada na L(0,1), está apresenta uma melhor discretização devido ao fato da sua extensão no eixo $ x $ ser menor. As distribuições L(0,0.01), L(0,0.25) e L(0,0.5) possuem um comportamento semelhante à N(0,1), enquanto as distribuições L(0,1.25) e L(0,1.5) semelhantes à L(0,1).% como pode ser visto no Apêndice~\ref{cap:anexoLin}

\begin{figure}[H]
	\centering\begin{subfigure}[b]{0.45\textwidth}
		\centering 
		\includegraphics[width=\linewidth]{./figuras/Linspace_normal_15_1000_0}
		\caption{}
		\label{fig:lin_norm15_data}
	\end{subfigure}
	\hfill
	\begin{subfigure}[b]{0.45\textwidth}
		\centering 
		\includegraphics[width=\linewidth]{./figuras/Linspace_normal_25_1000_0}
		\caption{}
		\label{fig:lin_norm25_data}
	\end{subfigure}
	\\
	\begin{subfigure}[b]{0.45\textwidth}
		\centering 
		\includegraphics[width=\linewidth]{./figuras/Linspace_normal_15_1000_25}
		\caption{}
		\label{fig:lin_norm15_data_out}
	\end{subfigure}
	\hfill
	\begin{subfigure}[b]{0.45\textwidth}
		\centering 
		\includegraphics[width=\linewidth]{./figuras/Linspace_normal_25_1000_25}
		\caption{}
		\label{fig:lin_norm25_data_out}
	\end{subfigure}
	\\
	\begin{subfigure}[b]{0.45\textwidth}
		\centering 
		\includegraphics[width=\linewidth]{./figuras/Linspace_lognormal_15_1000}
		\caption{}
		\label{fig:lin_lognorm15_data}
	\end{subfigure}
	\hfill
	\begin{subfigure}[b]{0.45\textwidth}
		\centering 
		\includegraphics[width=\linewidth]{./figuras/Linspace_lognormal_25_1000}
		\caption{}
		\label{fig:lin_lognorm25_data}
	\end{subfigure}
	\caption{Discretização com os dados gerados utilizando o método de \textit{Linspace}: (a) randN(0,1) com $N = 15$, (b) randN(0,1) com $N = 25$, (c) randN(0,1) com $N = 15$ e \textit{outlier} em $\pm 25$, (d) randN(0,1) com $N = 25$ e \textit{outlier} em $\pm 25$, (e) randL(0,1) com $ N = 15 $ e (f) randL(0,1) com $ N = 25 $.}
	\label{fig:lin_data}
\end{figure}

\subsection{Método \textit{CDFm}}

Pelo método \ac{CDFm} é esperado um acúmulo maior de pontos nas regiões em que a probabilidade é maior, como é ilustrado nas Figuras~\ref{fig:cdfnorm} e \ref{fig:cdf_data}, porém, é possível perceber que quanto mais rápida é a variação, melhor a sua discretização como é notado comparando as Figuras~\ref{fig:cdfnorm15} e \ref{fig:cdf_log15} e que a adição de \textit{outliers} não impacta tanto na discretização, mostrado nas Figuras~\ref{fig:cdf_norm15_data_out} e \ref{fig:cdf_lognorm25_data}. Esse mesmo padrão se repete tanto com os dados gerados mostrado na Figura~\ref{fig:cdf_data} quanto nas outras distribuições baseadas na função \textit{Lognormal}.%, como pode ser visto no Apêndice~\ref{cap:anexoCDFm}.

\begin{figure}[H]
	\centering
	\begin{subfigure}[b]{0.45\textwidth}
		\centering 
		\includegraphics[width=\linewidth]{./figuras/CDFm_normal_15}
		\caption{}
		\label{fig:cdfnorm15}
	\end{subfigure}
	\hfill
	\begin{subfigure}[b]{0.45\textwidth}
		\centering 
		\includegraphics[width=\linewidth]{./figuras/CDFm_normal_25}
		\caption{}
		\label{fig:cdfnorm25}
	\end{subfigure}
	
	
	\begin{subfigure}[b]{0.45\textwidth}
		\centering 
		\includegraphics[width=\linewidth]{./figuras/CDFm_lognormal_15}
		\caption{}
		\label{fig:cdf_log15}
	\end{subfigure}
	\hfill
	\begin{subfigure}[b]{0.45\textwidth}
		\centering 
		\includegraphics[width=\linewidth]{./figuras/CDFm_lognormal_25}
		\caption{}
		\label{fig:cdf_log25}
	\end{subfigure}
	
	\caption{Discretização utilizando o método de \textit{CDFm}: (a) N(0,1) com $N = 15$, (b) N(0,1) com $N = 25$, (c) L(0,1) com $N = 15$ e (d) L(0,1) com $N = 25$.}
	\label{fig:cdfnorm}
\end{figure}

\begin{figure}[H]
	\centering\begin{subfigure}[b]{0.45\textwidth}
		\centering 
		\includegraphics[width=\linewidth]{./figuras/CDFm_normal_15_1000_0}
		\caption{}
		\label{fig:cdf_norm15_data}
	\end{subfigure}
	\hfill
	\begin{subfigure}[b]{0.45\textwidth}
		\centering 
		\includegraphics[width=\linewidth]{./figuras/CDFm_normal_25_1000_0}
		\caption{}
		\label{fig:cdf_norm25_data}
	\end{subfigure}
	\\
	\begin{subfigure}[b]{0.45\textwidth}
		\centering 
		\includegraphics[width=\linewidth]{./figuras/CDFm_normal_15_1000_25}
		\caption{}
		\label{fig:cdf_norm15_data_out}
	\end{subfigure}
	\hfill
	\begin{subfigure}[b]{0.45\textwidth}
		\centering 
		\includegraphics[width=\linewidth]{./figuras/CDFm_normal_25_1000_25}
		\caption{}
		\label{fig:cdf_norm25_data_out}
	\end{subfigure}
	\\
	\begin{subfigure}[b]{0.45\textwidth}
		\centering 
		\includegraphics[width=\linewidth]{./figuras/CDFm_lognormal_15_1000_0}
		\caption{}
		\label{fig:cdf_lognorm15_data}
	\end{subfigure}
	\hfill
	\begin{subfigure}[b]{0.45\textwidth}
		\centering 
		\includegraphics[width=\linewidth]{./figuras/CDFm_lognormal_25_1000_0}
		\caption{}
		\label{fig:cdf_lognorm25_data}
	\end{subfigure}
	\caption{Discretização com os dados gerados utilizando o método \ac{CDFm}: (a) randN(0,1) com $N = 15$, (b) randN(0,1) com $N = 25$, (c) randN(0,1) com $N = 15$ e \textit{outlier} em $\pm 25$, (d) randN(0,1) com $N = 25$ e \textit{outlier} em $\pm 25$, (e) randL(0,1) com $ N = 15 $ e (f) randL(0,1) com $ N = 25 $.}
	\label{fig:cdf_data}
\end{figure}

Este método, se comparado ao \textit{Linspace}, apresenta um erro muito menor nas regiões de alta probabilidade. Em contrapartida, ele não deve ser indicado quando o objeto de estudo é a análise de eventos em regiões de baixa probabilidade, pois o mesmo teria que ter um número de pontos superior para o estudo dessas regiões.
%Vemos que a região de alta probabilidade é representada com um menor erro do que no método \textit{Linspace} mas, em contrapartida, a região de baixa probabilidade necessitaria de um número muito maior de pontos para possuir o mesmo erro do método anterior.

\subsection{Método \textit{PDFm}}

Como este método divide o eixo $ y $ de maneira uniforme, ele tende a penalizar menos as regiões de baixa probabilidade e que sejam simétricas se comparado com o método \ac{CDFm} como pode ser visto nas Figuras~\ref{fig:pdfnorm15}, \ref{fig:pdfnorm25}, \ref{fig:pdfm_norm15_data}, \ref{fig:pdfm_norm25_data}, \ref{fig:pdfm_norm15_data_out} e \ref{fig:pdfm_norm25_data_out}, já para o caso destas distribuições não serem simétricas, como mostra as Figuras~\ref{fig:pdflognorm15}, \ref{fig:pdflognorm15}, \ref{fig:pdfm_lognorm15_data} e \ref{fig:pdfm_lognorm25_data} a região de baixa probabilidade é mais afetada que o método anterior. Este padrão se repete para desvios padrões diferentes da distribuição Lognormal.%, que pode ser observado no Apêndice~\ref{cap:anexoPDFm}
%Este método, devido ao fato de dividir o eixo $ y $ de maneira uniforme, tende a penalizar mais regiões cujas inclinações são baixas, como a calda e o pico da distribuição sendo que, quanto mais lenta for a distribuição, mais essas regiões são penalizadas, não fazendo muito efeito em distribuições em que essa derivada é muito rápida. 

\begin{figure}[H]
	\centering
	\begin{subfigure}[b]{0.45\textwidth}
		\centering 
		\includegraphics[width=\linewidth]{./figuras/PDFm_normal_15}
		\caption{}
		\label{fig:pdfnorm15}
	\end{subfigure}
	\hfill
	\begin{subfigure}[b]{0.45\textwidth}
		\centering 
		\includegraphics[width=\linewidth]{./figuras/PDFm_normal_25}
		\caption{}
		\label{fig:pdfnorm25}
	\end{subfigure}
	
	\begin{subfigure}[b]{0.45\textwidth}
		\centering 
		\includegraphics[width=\linewidth]{./figuras/PDFm_lognormal_15_1}
		\caption{}
		\label{fig:pdflognorm15}
	\end{subfigure}
	\hfill
	\begin{subfigure}[b]{0.45\textwidth}
		\centering 
		\includegraphics[width=\linewidth]{./figuras/PDFm_lognormal_25_1}
		\caption{}
		\label{fig:pdflognorm25}
	\end{subfigure}
	
	\caption{Discretização utilizando o método de \textit{PDFm}: (a) N(0,1) com $N = 15$, (b) N(0,1) com $N = 25$, (c) L(0,1) com $N = 15$ e (d) L(0,1) com $N = 25$.}
	\label{fig:pdfmnorm}
\end{figure}

Deste modo, este método relaciona os pontos positivos da \ac{CDFm} para regiões de alta probabilidade sendo que seu erro para as regiões de baixa probabilidade é um pouco menor.
%Deste modo este método possui pontos positivos e negativos dos dois métodos anteriormente apresentados, \textit{Linspace} e \ac{CDFm} sendo mais indicado para distribuições simétricas, como a Gaussiana.

\begin{figure}[H]
	\centering\begin{subfigure}[b]{0.45\textwidth}
		\centering 
		\includegraphics[width=\linewidth]{./figuras/PDFm_normal_15_1000_0}
		\caption{}
		\label{fig:pdfm_norm15_data}
	\end{subfigure}
	\hfill
	\begin{subfigure}[b]{0.45\textwidth}
		\centering 
		\includegraphics[width=\linewidth]{./figuras/PDFm_normal_25_1000_0}
		\caption{}
		\label{fig:pdfm_norm25_data}
	\end{subfigure}
	\\
	\begin{subfigure}[b]{0.45\textwidth}
		\centering 
		\includegraphics[width=\linewidth]{./figuras/PDFm_normal_15_1000_25}
		\caption{}
		\label{fig:pdfm_norm15_data_out}
	\end{subfigure}
	\hfill
	\begin{subfigure}[b]{0.45\textwidth}
		\centering 
		\includegraphics[width=\linewidth]{./figuras/PDFm_normal_25_1000_25}
		\caption{}
		\label{fig:pdfm_norm25_data_out}
	\end{subfigure}
	\\
	\begin{subfigure}[b]{0.45\textwidth}
		\centering 
		\includegraphics[width=\linewidth]{./figuras/PDFm_lognormal_15_1000_0}
		\caption{}
		\label{fig:pdfm_lognorm15_data}
	\end{subfigure}
	\hfill
	\begin{subfigure}[b]{0.45\textwidth}
		\centering 
		\includegraphics[width=\linewidth]{./figuras/PDFm_lognormal_25_1000_0}
		\caption{}
		\label{fig:pdfm_lognorm25_data}
	\end{subfigure}
	\caption{Discretização com os dados gerados utilizando o método \ac{PDFm}: (a) randN(0,1) com $N = 15$, (b) randN(0,1) com $N = 25$, (c) randN(0,1) com $N = 15$ e \textit{outlier} em $\pm 25$, (d) randN(0,1) com $N = 25$ e \textit{outlier} em $\pm 25$, (e) randL(0,1) com $ N = 15 $ e (f) randL(0,1) com $ N = 25 $.}
	\label{fig:pdfm_data}
\end{figure}

%Para a distribuição Normal, este método apresenta um maior erro na região de alta probabilidade do que o método \ac{CDFm} mas nas regiões de baixa probabilidade o erro de estimação é menor do que o visto anteriormente, fazendo assim uma combinação dos métodos \textit{Linspace} e \textit{CDFm}.
\subsection{Método \textit{iPDF1}}

Este método faz a discretização baseada na primeira derivada, com isso é esperado uma melhor resolução nos pontos em que a variação da distribuição é maior, como pode ser verificado na Figura~\ref{fig:ipdfmnorm}.% e no Apêndice~\ref{cap:anexoiPDF1} em que este padrão também se repete. 

\begin{figure}[H]
	\centering
	\begin{subfigure}[b]{0.45\textwidth}
		\centering 
		\includegraphics[width=\linewidth]{./figuras/iPDF1_normal_15_1_0_0}
		\caption{}
		\label{fig:ipdfnorm15}
	\end{subfigure}
	\hfill
	\begin{subfigure}[b]{0.45\textwidth}
		\centering 
		\includegraphics[width=\linewidth]{./figuras/iPDF1_normal_25_1_0_0}
		\caption{}
		\label{fig:ipdfnorm25}
	\end{subfigure}
	
	\begin{subfigure}[b]{0.45\textwidth}
		\centering 
		\includegraphics[width=\linewidth]{./figuras/iPDF1_lognormal_15_1_0_0}
		\caption{}
		\label{fig:ipdflognorm15}
	\end{subfigure}
	\hfill
	\begin{subfigure}[b]{0.45\textwidth}
		\centering 
		\includegraphics[width=\linewidth]{./figuras/iPDF1_lognormal_25_1_0_0}
		\caption{}
		\label{fig:ipdflognorm25}
	\end{subfigure}
	
	\caption{Discretização utilizando o método de \textit{iPDF1}: (a) N(0,1) com $N = 15$, (b) N(0,1) com $N = 25$, (c) L(0,1) com $N = 15$ e (d) L(0,1) com $N = 25$.}
	\label{fig:ipdfmnorm}
\end{figure}

Já para os dados gerados, devido ao fato de sua derivada discreta ser ruidosa, sua CDF acaba sendo suavizada, fazendo com que este coloque mais pontos na região de baixa probabilidade, como pode ser visto na Figura~\ref{fig:ipdf1_data} e, em especial nas Figuras~\ref{fig:ipdf1_norm15_data_out} e \ref{fig:ipdf1_norm25_data_out} em que o erro na calda é baixo.

\begin{figure}[H]
	\centering\begin{subfigure}[b]{0.45\textwidth}
		\centering 
		\includegraphics[width=\linewidth]{./figuras/iPDF1_normal_15_1_1000_0}
		\caption{}
		\label{fig:ipdf1_norm15_data}
	\end{subfigure}
	\hfill
	\begin{subfigure}[b]{0.45\textwidth}
		\centering 
		\includegraphics[width=\linewidth]{./figuras/iPDF1_normal_25_1_1000_0}
		\caption{}
		\label{fig:ipdf1_norm25_data}
	\end{subfigure}
	\\
	\begin{subfigure}[b]{0.45\textwidth}
		\centering 
		\includegraphics[width=\linewidth]{./figuras/iPDF1_normal_15_1_1000_25}
		\caption{}
		\label{fig:ipdf1_norm15_data_out}
	\end{subfigure}
	\hfill
	\begin{subfigure}[b]{0.45\textwidth}
		\centering 
		\includegraphics[width=\linewidth]{./figuras/iPDF1_normal_25_1_1000_25}
		\caption{}
		\label{fig:ipdf1_norm25_data_out}
	\end{subfigure}
	\\
	\begin{subfigure}[b]{0.45\textwidth}
		\centering 
		\includegraphics[width=\linewidth]{./figuras/iPDF1_lognormal_15_1_1000_0}
		\caption{}
		\label{fig:ipdf1_lognorm15_data}
	\end{subfigure}
	\hfill
	\begin{subfigure}[b]{0.45\textwidth}
		\centering 
		\includegraphics[width=\linewidth]{./figuras/iPDF1_lognormal_25_1_1000_0}
		\caption{}
		\label{fig:ipdf1_lognorm25_data}
	\end{subfigure}
	\caption{Discretização com os dados gerados utilizando o método \ac{iPDF1}: (a) randN(0,1) com $N = 15$, (b) randN(0,1) com $N = 25$, (c) randN(0,1) com $N = 15$ e \textit{outlier} em $\pm 25$, (d) randN(0,1) com $N = 25$ e \textit{outlier} em $\pm 25$, (e) randL(0,1) com $ N = 15 $ e (f) randL(0,1) com $ N = 25 $.}
	\label{fig:ipdf1_data}
\end{figure}


\subsection{Método \textit{iPDF2}}

A \ac{iPDF2} se baseia na segunda derivada, com isso, é esperado que o número de pontos seja maior onde haja uma variação de sua derivada e menor nos pontos de inflexão, como é mostrado nas Figuras~\ref{fig:ipdf2norm15} e \ref{fig:ipdf2norm25}. O mesmo acontece para as Figuras~\ref{fig:ipdf2_lognorm15_data} e \ref{fig:ipdf2_lognorm25_data}, mas, como a taxa de variação é muito maior na porção esquerda da distribuição, a quantidade de pontos do lado direito fica reduzido, fazendo com que o erro de estimação aumente.

\begin{figure}[H]
	\centering
	\begin{subfigure}[b]{0.45\textwidth}
		\centering 
		\includegraphics[width=\linewidth]{./figuras/iPDF2_normal_15_1_0_0}
		\caption{}
		\label{fig:ipdf2norm15}
	\end{subfigure}
	\hfill
	\begin{subfigure}[b]{0.45\textwidth}
		\centering 
		\includegraphics[width=\linewidth]{./figuras/iPDF2_normal_25_1_0_0}
		\caption{}
		\label{fig:ipdf2norm25}
	\end{subfigure}
	
	\begin{subfigure}[b]{0.45\textwidth}
		\centering 
		\includegraphics[width=\linewidth]{./figuras/iPDF2_lognormal_25_1_0_0}
		\caption{}
		\label{fig:ipdf2lognorm15}
	\end{subfigure}
	\hfill
	\begin{subfigure}[b]{0.45\textwidth}
		\centering 
		\includegraphics[width=\linewidth]{./figuras/iPDF2_lognormal_25_1_0_0}
		\caption{}
		\label{fig:ipdf2lognorm25}
	\end{subfigure}
	
	\caption{Discretização utilizando o método de \textit{iPDF2}: (a) N(0,1) com $N = 15$, (b) N(0,1) com $N = 25$, (c) L(0,1) com $N = 15$ e (d) L(0,1) com $N = 25$.}
	\label{fig:ipdf2norm}
\end{figure}

Para os dados gerados, o mesmo problema ocorre no método \ac{iPDF1} só que com mais ruído, devido ao fato de ser a segunda derivada discreta, com isso a sua CDF é ainda mais suave, fazendo com que haja mais pontos na região de baixa probabilidade, diminuindo assim seu erro nessa região, mas aumentando na região onde a probabilidade é maior, conforme pode ser visto na Figura~\ref{fig:ipdf2_data}.

\begin{figure}[H]
	\centering\begin{subfigure}[b]{0.45\textwidth}
		\centering 
		\includegraphics[width=\linewidth]{./figuras/iPDF2_normal_15_1_1000_0}
		\caption{}
		\label{fig:ipdf2_norm15_data}
	\end{subfigure}
	\hfill
	\begin{subfigure}[b]{0.45\textwidth}
		\centering 
		\includegraphics[width=\linewidth]{./figuras/iPDF2_normal_25_1_1000_0}
		\caption{}
		\label{fig:ipdf2_norm25_data}
	\end{subfigure}
	\\
	\begin{subfigure}[b]{0.45\textwidth}
		\centering 
		\includegraphics[width=\linewidth]{./figuras/iPDF2_normal_15_1_1000_25}
		\caption{}
		\label{fig:ipdf2_norm15_data_out}
	\end{subfigure}
	\hfill
	\begin{subfigure}[b]{0.45\textwidth}
		\centering 
		\includegraphics[width=\linewidth]{./figuras/iPDF2_normal_25_1_1000_25}
		\caption{}
		\label{fig:ipdf2_norm25_data_out}
	\end{subfigure}
	\\
	\begin{subfigure}[b]{0.45\textwidth}
		\centering 
		\includegraphics[width=\linewidth]{./figuras/iPDF2_lognormal_15_1_1000_0}
		\caption{}
		\label{fig:ipdf2_lognorm15_data}
	\end{subfigure}
	\hfill
	\begin{subfigure}[b]{0.45\textwidth}
		\centering 
		\includegraphics[width=\linewidth]{./figuras/iPDF2_lognormal_25_1_1000_0}
		\caption{}
		\label{fig:ipdf2_lognorm25_data}
	\end{subfigure}
	\caption{Discretização com os dados gerados utilizando o método \ac{iPDF2}: (a) randN(0,1) com $N = 15$, (b) randN(0,1) com $N = 25$, (c) randN(0,1) com $N = 15$ e \textit{outlier} em $\pm 25$, (d) randN(0,1) com $N = 25$ e \textit{outlier} em $\pm 25$, (e) randL(0,1) com $ N = 15 $ e (f) randL(0,1) com $ N = 25 $.}
	\label{fig:ipdf2_data}
\end{figure}


Podemos ver que cada método se comporta melhor em uma determinada região, botando mais ou menos pontos nas regiões de interesse. Na seção seguinte será analisado o custo computacional de cada método proposto.% deste modo podemos analisar o custo computacional envolvido para cada método que será explicado na seção seguinte.
%Deste modo, esse estudo pode perceber as peculiaridades de cada método para um exemplo mais específico, e notar as regiões em que cada discretização se sai melhor, para assim poder ter um menor erro com número reduzido de pontos de estimação diminuindo o custo computacional da análise.

\section{Custo Computacional}

Os algoritmos de \textit{Kernel} são muito utilizados na literatura no contexto de análise de dados ou modelagem de dados, entretanto é sabido que esse método é computacionalmente mais lento em comparação com outros. Por isso muitos pesquisadores fazem uso de algoritmos que efetuam aproximações matemáticas no intuito de ganhar em custo computacional, chamados de \textit{FastKDE}, ou seja, existe um \textit{trade-off} entre estabilidade numérica e economia computacional. Entretanto, como já mencionado, o tempo de processamento esta diretamente ligado ao número de eventos da distribuição e ao número de pontos a serem estimados.

Portanto, no intuito de ilustrar a consequência de se aumentar o número de pontos de estimação a Figura \ref{fig:compKDE} apresenta o tempo de processamento de um algoritmo matricial de \textit{FastKDE} utilizado para a estimação de uma distribuição gaussiana $N(0,1)$ ao se variar o número de eventos e número de pontos de estimação.

\begin{figure}[!ht]
	\centering
	\includegraphics[width=0.8\linewidth]{./figuras/custocomp.png}\\
	\caption{Gráfico do tempo de processamento de um algoritmo de estimação de densidades baseado em KDE quando aumenta-se o número de eventos a serem estimados e o número de pontos de estimação.}
	\label{fig:compKDE}
\end{figure}

Pode-se observar que o tempo de processamento para $N = 1024$ é aproximadamente $75\%$ maior que para $N = 128$ quando o número de eventos é igual a $10^5$ e aproximadamente $67\%$ maior para número de eventos igual a $10^4$. Ou seja, um método de discretização capaz de apresentar o mesmo erro de estimação com menos pontos de estimação pode trazer benefícios importantes em ambientes de alta exigência.


\section{Ambiente de Análise}

Para analisar as diferenças entre a \ac{PDF} real e estimada ao longo de toda a extensão do eixo das abscissas, a área entre as duas \ac{PDF}s será usada como medida da estimação de erro. Além do mais, o eixo das abscissas foi dividida em várias regiões de mesmo tamanho, chamado \ac{RoI} \cite{ron1999art}. Essas regiões são compreendidas entre valores máximos e mínimos predefinidos do eixo horizontal. A Figura~\ref{fig:error} mostra este processo quando a abscissa é dividida em 20 regiões, todas compreendidas entre os valores $-4$ e $4$ do eixo $ x $.


\begin{figure}[!ht]
	\centering
	\includegraphics[width=0.6\linewidth]{./figuras/error1}\\
	\caption{Ilustração da medida de erro entre a PDF Real e a Estimada com 20 regiões de interesse.}
	\label{fig:error}
\end{figure}


A maneira que a \ac{RoI} é usada neste trabalho permitirá avaliar o erro de estimação em função de quatro diferentes parâmetros: Probabilidade; Eixo das abscissas; Primeira e Segunda Derivada. Para estimar os valores entre os pontos discretos, dois métodos de interpolação serão usados: interpolação pelo Vizinho Mais Próximo e Linear. 200 amostras serão usadas no processo de discretização. O erro de estimação tende a melhorar conforme o número de amostras aumenta mas sua característica geral não muda. Este último é a principal preocupação deste trabalho. 

%\chapter{Fatores Relevantes da Implementação de algoritmos baseados em KDE}\label{cap:algoritmo}

Neste capítulo, serão descritos alguns detalhes de construção do algoritmo de identificação de partículas desenvolvido no âmbito dessa tese e, além disso, serão ilustrados casos onde é possível otimizar a estimação via KDE. A estimação da densidade conjunta das variáveis discriminantes é central para o método de Verossimilhança e, devido a isto, este trabalho se concentrou neste tópico, a fim de indicar possíveis caminhos na definição de uma metodologia a ser adotada na implementação do KDE.

\section{Algoritmo}

O algoritmo de identificação de partículas baseado no método de verosimilhança pode ser resumido pelo diagrama de blocos da Figura \ref{fig:08}. Inicialmente são escolhidas as informações discriminantes, as variáveis do detector que serão utilizadas, como já mencionado na Seção \ref{sec:menu}, depois disso, é feita a estimação de densidade unidimensional e bidimensional (quando for o caso), com as PDF marginais faz-se o cálculo da PDF conjunta, aplica-se o discriminante e por fim retira-se a informação de classificação pelos índices \ac{SP}, dado pela Equação \ref{eq:07}, e \ac{AUC} \cite{bradley1997use}. Todo o processo é feito para cada região da Tabela \ref{tab:5T01} individualmente.

\begin{figure}[!ht]
	\begin{center}
		\includegraphics[width=0.9\linewidth]{./figuras/algoritmo1.png}
		\caption{Diagrama de blocos da utilização do método de Verossimilhança com as variáveis do \textit{Ringer}.}\label{fig:08}
	\end{center}
\end{figure}

\begin{equation}\label{eq:07}
    SP = \sqrt {\left( {\sqrt {VP \cdot FP} } \right) \cdot \left( {\frac{{VP + FP}}{2}} \right)}
\end{equation}
onde $VP$ e $FP$ são os valores de Verdadeiros Positivos e Falso Positivo, respectivamente, retirados da curva \ac{ROC}, que demonstra o desempenho de um sistema classificador binário.

\subsection{Validação Cruzada}

Todas as análises apresentadas aqui foram realizadas usando \ac{CV}. Segundo \cite{kohavi1995study}, a validação cruzada visa garantir a generalização do modelo proposto, evitando problemas como super-aprendizado ou super-ajuste. A implementação da \ac{CV} foi baseada em \textit{K-fold}, esse método consiste em separar o conjunto de dados em $k$ blocos e utilizar $k-1$ blocos para o conjunto de treinamento do algoritmo e o bloco restante para o conjunto de validação. Essa abordagem é repetida $k$ vezes alternando os blocos, como mostrado na Figura \ref{fig:09}. No caso desse trabalho foi escolhido o $k$ igual a 10, de acordo com o aconselhado por \cite{kohavi1995study}.

\begin{figure}[!ht]
	\begin{center}
		\includegraphics[width=0.45\linewidth]{./figuras/Kfold.png}
		\caption{Esquemático da validação cruzada baseada em \textit{k-fold}.}\label{fig:09}
	\end{center}
\end{figure}

\section{Alguns pontos para otimização do algoritmo de estimação de densidades}

Como citado anteriormente na Seção \ref{cap:rev} existem alguns tópicos de interesse na literatura com relação a otimização de algoritmos de estimação de densidades, alguns desses fora do escopo de estimação em si, mas não menos cruciais para o melhor desempenho dessa ferramenta. No caso dessa tese, visando otimização do processos de estimação e identificação de partículas, foram abordados os temas de Custo computacional, \textit{Outliers} e Discretização, que serão melhores discutidos nas seções subsequentes.

\subsection{Custo Computacional}

Os algoritmo de \textit{Kernel} são muito utilizados na literatura no contexto de análise de dados ou modelagem de dados, entretanto é sabido que esse método é computacionalmente mais lento em comparação com outros. Por isso muitos pesquisadores fazem uso de algoritmos que efetuam aproximações matemáticas no intuito de ganhar em custo computacional, chamados de \textit{FastKDE}, ou seja, existe um \textit{trade-off} entre estabilidade numérica e economia computacional. Entretanto, como o objetivo aqui é fazer uso da maior precisão numérica possível, foi desenvolvido nessa tese um algoritmo de KDE que efetua os cálculos de forma matricial, com isso há uma diminuição relevante de \textit{loopings} no algoritmo o que consequentemente colabora para a economia computacional, além disso, há ainda a possibilidade de dividir as matrizes criadas, quando o número de eventos for grande, efetuando o processamento em paralelo. Essas características fazem com que o KDE seja rápido e eficiente. 

Métodos de KDE comparados neste trabalho:

\begin{itemize}
\item KDE Naive (Banda Fixa) \cite{scott2015multivariate};
\item KDE Naive (Banda Variável) \cite{scott2015multivariate};
\item FastKDE (Banda Fixa);
\item FastKDE (Banda Variável);
\item KDE via Difussion (Banda Fixa) \cite{botev2010kernel};
\end{itemize}



Na Figura \ref{fig:compKDE} é apresentada uma comparação do método aqui desenvolvido com alguns métodos da literatura, em relação ao tempo de processamento e a Figura \ref{fig:erroKDE} mostra o erro de estimação de cada método. O primeiro gráfico varia em relação ao número de eventos e número de pontos de estimação, já o segundo varia somente em relação ao número de eventos, mantendo o número de pontos de estimação igual a 64, a distribuição estimada é a Gaussiana $N(0,1)$.

\begin{figure}[!ht]
	\centering
	\includegraphics[width=12cm]{./textuais/desenvolvimento/figuras/compKDE.png}\\
	\caption{Comparação entre diferentes métodos de FastKDE em relação ao tempo de processamento, para uma distribuição Gaussiana.}
	\label{fig:compKDE}
\end{figure}

Pode-se notar que o \textit{FastKDE} implementado nesse trabalho apresenta o menor tempo de processamento em relação aos métodos que fazem uso de largura de banda variável e mesmo assim apresenta erro de estimação baixo. Para o caso ilustrado as estimações com largura de banda fixa atingem erros menores ao serem comparadas com os métodos de largura de banda variável, uma vez que esses são ótimos para uma distribuição Gaussiana \cite{scott2015multivariate}.

\begin{figure}[!ht]
	\centering
	\includegraphics[width=12cm]{./textuais/desenvolvimento/figuras/erroKDE.png}\\
	\caption{Comparação entre diferentes métodos de KDE em relação ao erro de estimação, para uma distribuição Gaussiana.}
	\label{fig:erroKDE}
\end{figure}

Já quando o objetivo é estimar uma distribuição \textit{Lognormal} $Logn(0,0.8)$ nota-se que os estimadores que utilizam largura de banda variável atingem valores de erros menores que os outros, como mostrado na Figura \ref{fig:erroKDElog}. E o tempo de processamento dos métodos para a distribuição \textit{lognormal} pode ser visto na Figura \ref{fig:compKDElog}.

\begin{figure}[!ht]
	\centering
	\includegraphics[width=12cm]{./textuais/desenvolvimento/figuras/compKDElog.png}\\
	\caption{Comparação entre diferentes métodos de FastKDE em relação ao tempo de processamento, para uma distribuição \textit{Lognormal}.}
	\label{fig:compKDElog}
\end{figure}

\begin{figure}[!ht]
	\centering
	\includegraphics[width=12cm]{./textuais/desenvolvimento/figuras/erroKDElog.png}\\
	\caption{Comparação entre diferentes métodos de KDE em relação ao erro de estimação, para uma distribuição \textit{Lognormal}.}
	\label{fig:erroKDElog}
\end{figure}

\clearpage

\subsection{Descontinuidades e \textit{Outliers}}

O pré-processamento dos dados é utilizado para a remoção de descontinuidades e \textit{outliers} presentes nos histogramas das variáveis discriminantes. Sendo o primeiro um efeito que ocorre, principalmente, por indeterminações matemáticas no processo de cálculo das probabilidades, como por exemplo, divisão por zero, já os \textit{outliers} são eventos que ocorrem na região de baixa probabilidade, distantes da região central da PDF. Exemplos desses efeitos são mostrados no caso representativo da Figura \ref{fig:5T06}.

  \begin{figure}[h]
	\centering
	\includegraphics[width=10cm]{./textuais/algoritmo/figuras/explicacaooutilers.png}\\
	\caption{Gráfico da distribuição de uma variável, exemplificando os Valores Extremos (\emph{Outliers}) e Descontinuidades (\emph{Discontinuites}).}
	\label{fig:5T06}
\end{figure}


\subsection{Discretização}

Atualmente os algoritmos de estimação de densidade via KDE utilizam a discretização com espaçamento uniforme, ou seja, depois de feita a estimação faz-se necessário escolher alguns pontos que serão armazenados para representar aquela distribuição, nesse momento monta-se uma grade de pontos distribuídos uniformemente no eixo das abscissas. Entretanto, fazer o uso desse espaçamento para variáveis que não são gaussianas pode contribuir para uma estimação não-otimizada das distribuições. Em \cite{costa2017study} é feito um estudo sobre a utilização de 4 métodos de escolha da discretização da grade, sendo um deles baseado na \ac{CDF}, que podem tornar a estimação de densidades menos suscetível a \textit{outliers} e distribuições com longas caudas. Na Figura \ref{fig:linspace}, é ilustrado o funcionamento do método linear (padrão na literatura) e do método baseado na \ac{CDF}.

\begin{figure}[H]
		\centering
		\includegraphics[width=7cm]{./textuais/algoritmo/figuras/2.pdf}
        \includegraphics[width=7.5cm]{./textuais/algoritmo/figuras/1.pdf}
		\caption{Caso representativo do método de discretização baseado no espaçamento linear (Esquerda) e baseado na \ac{CDF} (Direita) aplicado a uma distribuição Gaussiana.}
		\label{fig:linspace}
\end{figure}

Essa ferramenta está sendo avaliada em um ambiente controlado, utilizando de distribuições parametrizadas, para ter conhecimento de todas as suas características e, então, será testada no ambiente de física de altas energias.

\section{Estimação de densidades das regiões de interesse} \label{sec:difest}

A implementação do algoritmo baseia-se nas formulações matemáticas descritas na Seção~\ref{sec:kdeuni}. Nessa abordagem, é considerado um KDE com banda variável otimizado através do parâmetro $\lambda$, de acordo com a otimização apresentada na Seção~\ref{sec:bandavariavel}, restando utilizar o histograma, com a binagem escolhida baseada no algoritmo proposto por \cite{shimazaki2007method}, para calcular o valor de $f\left( x \right)$ de cada evento, como mostrado na equação~\ref{eq:54}.

Computacionalmente, no processo de estimação das densidades, deve-se escolher o número de pontos a ser usado para a sua representação discreta. Adicionando esse número aos parâmetros mencionados no parágrafo anterior, as PDF podem ser construídas através do método de KDE. O algoritmo implementado foi programado para gerar 2000 pontos para cada variável, de maneira a garantir uma boa representação de suas respectivas distribuições. Os valores das variáveis gerados pelos eventos de colisão são, então, interpolados linearmente a partir dos pontos da PDF discreta.

As Figuras~\ref{fig:9T20} até~\ref{fig:9T23} mostram alguns casos representativos das distribuição a serem estimadas pelo algoritmo de KDE na Região 4 (Esquerda) e Região 15 (Direita), para o conjunto de sinal (denominado \emph{electron}, por ser formado por elétrons isolados) e para o conjunto de ruído de fundo (denominado \emph{Jet}, pela predominância de jatos). A escolha da melhor binagem para visualização do \ac{ASH} foi feita utilizando o método de Freedman-Diaconis \cite{freedman1981histogram}.

\begin{figure}[H]
	\centering
	\includegraphics[width=7cm]{./textuais/algoritmo/figuras/DISTKERNELET3ETA0VAR1KF1.png}
    \includegraphics[width=7cm]{./textuais/algoritmo/figuras/DISTKERNELET4ETA2VAR1KF1.png}\\
	\caption{Gráficos das distribuições de elétron e jato do anel 1. (Esquerda) Região 4 e (Direita) Região 15.}
	\label{fig:9T20}
\end{figure}

\begin{figure}[H]
	\centering
	\includegraphics[width=7cm]{./textuais/algoritmo/figuras/DISTKERNELET3ETA0VAR10KF1.png}
    \includegraphics[width=7cm]{./textuais/algoritmo/figuras/DISTKERNELET4ETA2VAR10KF1.png}\\
	\caption{Gráficos das distribuições de elétron e jato do anel 10. (Esquerda) Região 4 e (Direita) Região 15.}
	\label{fig:9T21}
\end{figure}

\begin{figure}[H]
	\centering
	\includegraphics[width=7cm]{./textuais/algoritmo/figuras/DISTKERNELET3ETA0VAR20KF1.png}
    \includegraphics[width=7cm]{./textuais/algoritmo/figuras/DISTKERNELET4ETA2VAR20KF1.png}\\
	\caption{Gráficos das distribuições de elétron e jato do anel 20. (Esquerda) Região 4 e (Direita) Região 15.}
	\label{fig:9T22}
\end{figure}

\begin{figure}[H]
	\centering
	\includegraphics[width=7cm]{./textuais/algoritmo/figuras/DISTKERNELET3ETA0VAR73KF1.png}
    \includegraphics[width=7cm]{./textuais/algoritmo/figuras/DISTKERNELET4ETA2VAR73KF1.png}\\
	\caption{Gráficos das distribuições de elétron e jato do anel 73. (Esquerda) Região 4 e (Direita) Região 15.}
	\label{fig:9T23}
\end{figure}

De acordo com as Figuras~\ref{fig:9T20} até~\ref{fig:9T23} temos que as distribuições das variáveis discriminantes provenientes da Região 4 tem transições mais suaves, não possuem mais de um cume, tem derivadas e caudas menores em comparação com as distribuições obtidas da Região 15. Esse detalhe faz com que a estimação das PDF da Região 15 seja mais complexa, demandando algoritmos mais robustos e com características particulares, além disso a Região 15 contém 6 vezes menos estatística do que a Região 4, de acordo com a Tabela \ref{tab:5T01}, isso significa que existe a exigência de uma performance ótima mesmo com pouca estatística. Tais dificuldades abrem caminho para a pesquisa na direção de otimização de algoritmos de estimação de densidades baseado em núcleo.

Além disso, as variáveis provenientes de sensores similares, calorímetros por exemplo, são mais propensas a apresentar correlação mútua, como é o caso das variáveis de \textit{Ringer}, como pode ser visto nas Figuras \ref{fig:9T24} e \ref{fig:9T25}, isso faz com que o uso de todas as 100 variáveis ao mesmo tempo ocasione um erro devido a simplificação adotada do método de verossimilhança, que faz a suposição de independência entre as variáveis.

\begin{figure}[H]
	\centering
	\includegraphics[width=12cm]{./textuais/algoritmo/figuras/plotCorrElectron.png}\\
	\caption{Gráficos da correlação entre as variáveis do \textit{Ringer} para elétron.}
	\label{fig:9T24}
\end{figure}

\begin{figure}[H]
	\centering
    \includegraphics[width=12cm]{./textuais/algoritmo/figuras/plotCorrJet.png}\\
	\caption{Gráficos da correlação entre as variáveis do \textit{Ringer} para jato.}
	\label{fig:9T25}
\end{figure}

Portanto é necessário o uso de uma ferramenta para escolher quais anéis serão utilizados em cada região, no intuito de viabilizar a implementação desse algoritmo nas variáveis de \textit{Ringer}. A escolha pelo uso de uma técnica de busca utilizada para achar soluções aproximadas em problemas de otimização foi feita baseada na estrutura do problema em questão.

\section{Algoritmo Genético}

O \ac{GA} baseia-se em uma codificação do conjunto das possíveis soluções, apresenta os resultados como uma população de soluções, não necessita de nenhum conhecimento derivado do problema, mas somente de uma forma de avaliação do resultado (função a ser minimizada) e usa transições probabilísticas ao invés de regras determinísticas, tais características justificam a escolha dessa ferramenta para a escolha das variáveis discriminantes a serem utilizadas pelo algoritmo de identificação de elétrons.

O \ac{GA} utilizado embasa-se na implementação de \cite{haupt1998practical}, que é um Algoritmo Genético Binário e foi configurado alguns parâmetros para se adequar a realidade do problema em questão. Segue abaixo o \textit{setup} utilizado pelo \ac{GA}:

\begin{itemize}
\item Número máximo de iterações = $100$;
\item Tamanho da população = $60$ indivíduos;
\item Taxa de mutação = $2\%$;
\item Fração da população mantida entre cada iteração = $50\%$;
\item Número de bits no cromossomo = $100$;
\item População inicial = $1$;
\item Função custo = Minimização dos parâmetros SP ou AUC da classificação.
\end{itemize}

Esses parâmetros possibilitando que o \ac{GA} seja capaz de encontrar um dos melhores conjunto de anéis com a menor quantidade de anéis possível, visto que sua população inicial é igual a $1$, sua taxa de mutação é suficiente para escapar de mínimos locais, mas não grande ao ponto de tornar o algoritmo completamente aleatório. Esse método é utilizado no conjunto de treinamento para decidir quais anéis serão utilizados no conjunto de validação.

\section{Principais diferenças em relação ao método utilizado pela Colaboração ATLAS}

Como visto nas seções precedentes, a busca por uma otimização do algoritmo baseado no método de Verossimilhança a partir da estimação de densidades univariadas desenvolvida neste trabalho levaram a uma implementação ligeiramente diferente daquela em uso atualmente pela Colaboração. A Tabela~\ref{tab:5T02} resume as principais diferenças encontradas.

\begin{table}[H]
  \centering
  \caption{Tabela de diferenças entre a \ac{LH} da colaboração \cite{atlasdescription} e o algoritmo implementado nessa dissertação.}\label{tab:5T02}
\begin{tabular}{c|c|c}
  % after \\: \hline or \cline{col1-col2} \cline{col3-col4} ...
  & LH Colaboração Atlas & LH desse trabalho  \\ \hline \hline
  Método & Verossimilhança & Verossimilhança \\
  Variáveis & 13 & 106 \\
  Cortes Adicionais & Todos & Nenhum \\
  Interpolação & * & Linear \\
  Extrapolação & Vizinho mais próximo & Exponencial \\
  PDF & KDE & MKDE \\
  Regiões em $\eta$ e $E_t$ & 9x6 & 4x5 \\
  Pontos na PDF & 520** & 2000 \\
\end{tabular}

\begin{flushleft}
*A nota \cite{atlasdescription} não deixa clara a forma de interpolação utilizada.
**Para a variável TRT, utilizaram-se 62 pontos.
\end{flushleft}
\end{table}


Nesse capítulo foram apresentados os principais testes de definições relacionados ao algoritmo de classificação de partículas via método de verossimilhança com abordagem univariada utilizando as variáveis de \textit{Ringer}, bem como apontado as otimizações  que podem ser realizadas nele no intuito de se obter um melhor desempenho na classificação dos eventos. No Capítulo~\ref{cap:Resultados} serão apresentados os resultados obtidos com a aplicação desses métodos nos dados de validação.

%\chapter{Resultados} \label{cap:resultados}

Nessa sessão, os resultados serão dados em termos da área medida entre a diferença da \ac{PDF} real e a estimada, como previamente mencionado. Nas figuras que seguem , os valores serão mostrados no eixo vertical, nomeado de \textit{Erro}. O eixo horizontal mostra quatro perspectivas diferentes: Probabilidade, Eixo $ x $ (que seria os valores aleatórios da variável), Primeira derivada, e segunda derivada. Essas perspectivas diferentes irão permitir um melhor entendimento das características de cada método. Cada gráfico será completado com 100 pontos de dados, cada um representando o erro medido encontrado em cada \ac{RoI} (Contudo, nessa sessão, serão considerados 100 \ac{RoI} ao invés de 20 como mostrado na Figura~\ref{fig:error}). Essa sessão será será dividida em três análises diferentes. Sessão~\ref{cap:interp_neares} analisa a estimação de erro quando a interpolação pelo vizinho mais próximo é usada; Sessão~\ref{cap:interp_lin} avalia para a interpolação linear; e a Sessão~\ref{cap:interp_neares} insere problemas de \textit{outliers}. Quando \textit{outliers} são gerados, o desempenho de alguns métodos de discretização podem ser altamente degradados em comparação com outros, sendo uma questão importante a ser analisada.

\section{Estimação de erro pela interpolação do vizinho mais próximo} \label{cap:interp_neares}
A interpolação pelo vizinho mais próximo basicamente atribui o valor vizinho mais próximo ao valor da probabilidade da variável aleatória que será estimada. Portanto, o erro de estimação será proporcional à sua distância da amostra mais próxima. Tal método de interpolação produz um erro diretamente proporcional à primeira derivada \cite{gurevich1966integral}. Analisando as Figura~\ref{fig:12a} e \ref{fig:12b} pode-se inferir que: 

\begin{figure}[H]
	\centering
	\begin{subfigure}[b]{0.45\textwidth}
		\centering 
		\includegraphics[width=\textwidth]{./figuras/error_normal_nearest_Probabilidade}
		\caption{}
		\label{fig:12a}
	\end{subfigure}
	\hfill
	~ %add desired spacing between images, e. g. ~, \quad, \qquad, \hfill etc. 
	%(or a blank line to force the subfigure onto a new line)
	\begin{subfigure}[b]{0.45\textwidth}
		\centering 
		\includegraphics[width=\textwidth]{./figuras/error_normal_nearest_X}
		\caption{}
		\label{fig:12b}
	\end{subfigure}
	%\vskip\baselineskip
	~ %add desired spacing between images, e. g. ~, \quad, \qquad, \hfill etc. 
	%(or a blank line to force the subfigure onto a new line)
	\begin{subfigure}[b]{0.45\textwidth}
		\centering 
		\includegraphics[width=\textwidth]{./figuras/error_normal_nearest_Primeira_Derivada.png}
		\caption{}
		\label{fig:12c}
	\end{subfigure}
	\hfill
	\begin{subfigure}[b]{0.45\textwidth}
		\centering 
		\includegraphics[width=\textwidth]{./figuras/error_normal_nearest_Segunda_Derivada.png}
		\caption{}
		\label{fig:12d}
	\end{subfigure}
	\caption{Caso representativo com 200 pontos, 100 \ac{RoI} e usando a interpolação pelo vizinho mais próximo.}
	\label{fig:12}
\end{figure}

\begin{description}
	\item[Linspace] erro de estimação aumenta com a 1ª derivada da função que está sendo estimada;
	\item[CDFm] erro de estimativa diminui com o aumento da probabilidade; 
	\item[PDFm e iPDF1] tendem a equalizar o erro de estimativa aumentando a densidade de pontos discretos nas regiões de derivadas mais altas; 
	\item[iPDF2] diminui o erro com o aumento da probabilidade, no entanto, apresenta um aumento de erro próximo ao ponto de inflexão.
\end{description}  

Na Figura~\ref{fig:12c} é possível perceber que:
 \begin{description}
	\item[Linspace] apresenta um aumento do erro de estimação diretamente proporcional ao valor da 1ª derivada;
	\item[CDFm] apresenta maior densidade de pontos na região de alta probabilidade e poucos pontos na região de baixa probabilidade. No entanto, do ponto de vista da 1ª derivada, o erro de estimação oscila entre valores altos e baixos;
	\item[PDFm e iPDF1] tendem a manter constante o valor do erro de estimação, uma vez que se coloca mais pontos em regiões com derivadas maiores, onde o erro do método de discretização pelo vizinho mais próximo é maior. Porém, como mencionado anteriormente, as caudas apresentam os menores erros, o que justifica a diminuição do erro nas regiões de baixa derivada;
	\item[iPDF2] aumenta o erro de estimação de acordo com o aumento da 1ª derivada, uma vez que tais regiões tendem a receber menos pontos estimados.
\end{description}  

Finalmente, quando a Figura \ref{fig:12d} é observada (lembrando que valores baixos de 2ª derivada representam regiões intercaladas de baixa probabilidade e inflexão, de modo que, estas são as situações onde o erro causado pela interpolação é menor) ela pode ser inferida que:

\begin{description}
	\item[Linspace e iPDF2] exibem comportamento similar, aumentam o erro diretamente proporcional à 2ª derivada, porém, ao se aproximarem da região de alta probabilidade, o erro tende a cair;
	\item[CDFm] tende a diminuir em erro a medida que a segunda derivada aumenta;
	\item[PDFm e iPDF1] não parecem sofrer com a variação da 2ª derivada, exceto em regiões de valores baixos, onde o erro flutua.
\end{description}

Podemos também realizar uma análise de desempenho verificando a equivalência do erro de estimação para quando o número estimado varia, conforme mostra a Figura~\ref{fig:errorplotnearest}.

\begin{figure}[H]
	\centering
	\begin{subfigure}[b]{0.45\textwidth}
		\centering 
		\includegraphics[width=\textwidth]{./figuras/ERRORPLOT_L1_TRUE_NORMAL_NEAREST_00}
		\caption{}
		\label{fig:errornormnearest}
	\end{subfigure}
	\hfill
	~ %add desired spacing between images, e. g. ~, \quad, \qquad, \hfill etc. 
	%(or a blank line to force the subfigure onto a new line)
	\begin{subfigure}[b]{0.45\textwidth}
		\centering 
		\includegraphics[width=\textwidth]{./figuras/ERRORPLOT_L1_TRUE_LOGNORMAL_NEAREST_00}
		\caption{}
		\label{fig:errorlognearest}
	\end{subfigure}
	
	\caption{Erro total de estimação para a interpolação pelo vizinho mais próximo variando-se o números de pontos a se estimar: (a) N(0,1) e (b) L(0,1)}
	\label{fig:errorplotnearest}
\end{figure}

Podemos perceber que na Figura~\ref{fig:errornormnearest}, devido ao fato de possuir uma derivada mais lenta, o método \textit{Linspace} é o que possui o menor erro, juntamente com os métodos \ac{PDFm} e \ac{iPDF1}, por outro lado, o método \ac{CDFm} é o que possui o pior desempenho, fazendo com que se necessite de mais pontos de estimação para possuir o mesmo erro total que os outros métodos. Já quando a distribuição possui uma variação maior, a cena se inverte, como é mostrada na Figura~\ref{fig:errorlognearest} em que a \ac{CDFm} é a que possui o menor erro, sendo necessário, por exemplo possuir apenas 120 pontos de estimação enquanto o \textit{Linspace} necessita de 300 pontos para manter o mesmo erro.


\section{Estimação de erro pela interpolação linear} \label{cap:interp_lin}

O método de interpolação linear naturalmente tende a apresentar erros de estimação mais altos em regiões com a 2ª derivada maior se os pontos discretos forem distribuídos uniformemente ao longo do eixo horizontal, como pode ser notado pelo método \textit{Linspace} da Figura \ref{fig:11d}. Adicionalmente, analisando as Figuras \ref{fig:11a} e \ref{fig:11b} é possível observar o seguinte:

\begin{description}
	\item[Linspace] erro de estimação diminui em regiões de baixa probabilidade e perto dos pontos de inflexão;
	\item[CDFm] erro de estimação diminui com o aumento da probabilidade, e nos pontos de inflexão há uma melhoria ainda maior;
	\item [PDFm e iPDF1] erro aumenta em regiões de baixa e alta probabilidade e diminui próximo aos pontos de inflexão;
	\item[iPDF2] apresenta menor densidade de pontos em regiões de baixa probabilidade e regiões de inflexão, levando a um comportamento inverso quando comparado ao método \textit{Linspace}.
\end{description}   

\begin{figure}[H]
	\centering
	\begin{subfigure}[b]{0.45\textwidth}
		\centering 
		\includegraphics[width=\textwidth]{./figuras/error_normal_linear_Probabilidade}
		\caption{}
		\label{fig:11a}
	\end{subfigure}
	\hfill
	~ %add desired spacing between images, e. g. ~, \quad, \qquad, \hfill etc. 
	%(or a blank line to force the subfigure onto a new line)
	\begin{subfigure}[b]{0.45\textwidth}
		\centering 
		\includegraphics[width=\textwidth]{./figuras/error_normal_linear_X}
		\caption{}
		\label{fig:11b}
	\end{subfigure}
	%\vskip\baselineskip
	~ %add desired spacing between images, e. g. ~, \quad, \qquad, \hfill etc. 
	%(or a blank line to force the subfigure onto a new line)
	\begin{subfigure}[b]{0.45\textwidth}
		\centering 
		\includegraphics[width=\textwidth]{./figuras/error_normal_linear_Primeira_Derivada.png}
		\caption{}
		\label{fig:11c}
	\end{subfigure}
	\hfill
	\begin{subfigure}[b]{0.45\textwidth}
		\centering 
		\includegraphics[width=\textwidth]{./figuras/error_normal_linear_Segunda_Derivada.png}
		\caption{}
		\label{fig:11d}
	\end{subfigure}
	
	\caption{Caso representativo com 200 pontos, 100 \ac{RoI} e usando a interpolação linear.}\label{fig:11}
\end{figure}

Avaliando a Figura \ref{fig:11c} é possível confirmar que os métodos \textit{Linspace} e \ac{iPDF2} têm comportamentos opostos. Os outros métodos se comportam de maneira semelhante, pois reduzem o erro de estimação quando a 1ª derivada aumenta.
Na Figura \ref{fig:11d}, é possível notar que o método \textit{Linspace} mostra a mesma tendência em relação à 2ª derivada para o caso da interpolação linear da apresentada em relação à 1ª derivada para o caso da interpolação pelo vizinho mais próximo.
A flutuação de erro observada no método \ac{CDFm} é causada pela variação entre as regiões de alta e baixa probabilidade ao longo do eixo da 2ª derivada, no entanto, uma vez que as regiões de alta probabilidade estão associadas à regiões de segunda derivada alta, esta oscilação tende a desaparecer com o aumento desta derivada.
Isso também explica as flutuações nos métodos \ac{PDFm} e \ac{iPDF1}, porém esses métodos apresentam desempenho degradado em regiões de alta probabilidade. Finalmente, o método \ac{iPDF2} mantém o comportamento inverso ao método \textit{Linspace}.

Ao mudarmos para a interpolação linear, é possível perceber que o erro diminui de maneira considerável, como pode ser visto na Figura~\ref{fig:errorplotlin} em que para a distribuição Normal, o método \textit{Linspace} se sobressai a todos os outros e, como na interpolação pelo vizinho mais próximo o método \ac{CDFm} é o pior, como ilustra a Figura~\ref{fig:errornormlin}, já para a distribuição Lognormal, o método \ac{CDFm} se sobressai até os 1000 pontos de estimação, após o método \textit{Linspace} se torna mais eficiente como é ilustrado na Figura~\ref{fig:errorloglin} devido ao fato de ser o único método que estima com maior precisão as regiões de baixa probabilidade.

\begin{figure}[H]
	\centering
	\begin{subfigure}[b]{0.45\textwidth}
		\centering 
		\includegraphics[width=\textwidth]{./figuras/ERRORPLOT_L1_TRUE_NORMAL_LINEAR_00}
		\caption{}
		\label{fig:errornormlin}
	\end{subfigure}
	\hfill
	~ %add desired spacing between images, e. g. ~, \quad, \qquad, \hfill etc. 
	%(or a blank line to force the subfigure onto a new line)
	\begin{subfigure}[b]{0.45\textwidth}
		\centering 
		\includegraphics[width=\textwidth]{./figuras/ERRORPLOT_L1_TRUE_LOGNORMAL_LINEAR_00}
		\caption{}
		\label{fig:errorloglin}
	\end{subfigure}
	
	\caption{Erro total de estimação para a interpolação linear variando-se o números de pontos a se estimar: (a) N(0,1) e (b) L(0,1)}
	\label{fig:errorplotlin}
\end{figure}

\section{Estimação de erro considerando \textit{outliers}} \label{cap:erro_out}

A fim de verificar o comportamento dos métodos de discretização em uma realidade onde existem conjuntos de dados com \textit{outliers}, como é comum em experimentos reais, \textit{outliers} foram inseridos nos dados gerados. As posições \textit{outliers} foram varridas até o valor de 50.
O problema dos \textit{outliers} pode ser visto de outra perspectiva, relacionado à definição dos limites do eixo horizontal, que é geralmente um pré-requisito para aplicar algoritmos de estimativa de \ac{PDF}.
Além disso, o número de pontos estimados foi varrido para 1000. Esta análise pode ser vista na Figura~\ref{fig:13}.

\begin{figure}[H]
	\centering
	\begin{subfigure}[b]{0.45\textwidth}
		\centering 
		\includegraphics[width=\textwidth]{./figuras/erro3d_linspace}
		\caption{}
		\label{fig:13a}
	\end{subfigure}
	\hfill
	\begin{subfigure}[b]{0.45\textwidth}
		\centering 
		\includegraphics[width=\textwidth]{./figuras/figure13b.pdf}
		\caption{}
		\label{fig:13b}
	\end{subfigure}
	\caption{Análise de \textit{outlier} usando 100 Rois e interpolação linear.}
	\label{fig:13}
\end{figure}

A figura \ref{fig:13a} mostra que o método \textit{Linspace} aumenta consideravelmente seu erro de estimação quando \textit{outliers} são inseridos; quanto mais distantes os \textit{outliers}, maior o erro. Este efeito é mitigado pelo aumento do número de pontos estimados.
A Figura~\ref{fig:13b} mostra que os métodos propostos são menos sensíveis aos \textit{outliers} (ou à escolha dos limites do eixo horizontal). As tabelas \ref{tab:near} e \ref{tab:lin} mostram o erro médio dos métodos para três posições de \textit{outliers} (0, 20 e 50) e 100 pontos de estimação), para interpolações do vizinho mais próximo e linear, respectivamente.

\begin{table}[H]
	\centering
	\caption{Erro de estimativa média usando a distribuição normal e interpolação do vizinho mais próximo com 100 pontos de estimação.}
	\label{tab:near}
	\begin{tabular}{llllll}
		\hline
		\multicolumn{6}{c}{\textbf{Média}}                                                                                             \\ \hline
		\multicolumn{1}{l|}{\textbf{\#Outlier}} & \textbf{Linspace} & \textbf{CDFm} & \textbf{PDFm} & \textbf{iPDF1} & \textbf{iPDF2} \\ \hline
		\multicolumn{1}{l|}{\textbf{0}}         & 8.02e-5           & 1.96e-4       & 8.26e-5       & 8.19e-5        & 1.20e-4        \\
		\multicolumn{1}{l|}{\textbf{20}}        & 4.82e-4           & 1.96e-4       & 1.14e-4       & 9.22e-5        & 1.25e-4        \\
		\multicolumn{1}{l|}{\textbf{50}}        & 1.09e-3           & 1.96e-4       & 1.14e-4       & 9.22e-5        & 1.25e-4        \\ \hline
	\end{tabular}
	
\end{table}

\begin{table}[H]
	\centering
	\caption{Erro de estimação média usando a distribuição normal e interpolação linear com 100 pontos de estimação.}
	\label{tab:lin}
	\begin{tabular}{llllll}
		\hline
		\multicolumn{6}{c}{\textbf{Média}}                                                                                             \\ \hline
		\multicolumn{1}{l|}{\textbf{\#Outlier}} & \textbf{Linspace} & \textbf{CDFm} & \textbf{PDFm} & \textbf{iPDF1} & \textbf{iPDF2} \\ \hline
		\multicolumn{1}{l|}{\textbf{0}}         & 1.30e-6           & 1.01e-4       & 2.14e-5       & 2.07e-5        & 4.97e-6        \\
		\multicolumn{1}{l|}{\textbf{20}}        & 4.66e-5           & 1.01e-4       & 5.61e-5       & 3.01e-5        & 8.35e-6        \\
		\multicolumn{1}{l|}{\textbf{50}}        & 2.31e-4           & 1.01e-4       & 6.38e-5       & 3.00e-5        & 8.35e-6      \\
		\hline
	\end{tabular}
	\
\end{table}

Para a interpolação do vizinho mais próximo, o método \ac{iPDF1} apresentou o melhor desempenho enquanto para a interpolação Linear, o \ac{iPDF2} foi o melhor se o erro de estimação e a sensibilidade de \textit{outliers} forem considerados. O \ac{CDFm} mostrou-se praticamente imune a \textit{outliers}. Quando nenhum valor discrepante está presente, o Linspace atinge o melhor resultado seguido de perto pelo \ac{PDFm} e \ac{iPDF1} para o caso do vizinho mais próximo e pelo \ac{iPDF2} para o caso Linear. No entanto, é influenciado pela escolha arbitrária de 99.99\% da área como limites do eixo horizontal padrão para caracterizar a ausência de \textit{outliers}. 
A figura~\ref{fig:Error_out} mostra como o erro varia conforme é aumentado a quantidade de pontos de estimação, sendo que, como é mostrado na Figura~\ref{fig:error_norm_near_50}, há uma mudança na performance de cada método, fazendo com que alguns se sobressaiam sobre outros conforme é aumentado o número de pontos, para este caso, utilizando a interpolação pelo vizinho mais próximo, os métodos \ac{iPDF1} e \ac{iPDF2} foram os que tiveram o menor erro até os 5000 pontos estimados. Já para a interpolação Linear, mostrada na Figura~\ref{fig:error_norm_lin_50} os métodos \ac{iPDF1}, \ac{iPDF2} e \ac{CDFm} estagnam por volta do 500º ponto de estimação, não alterando mais tanto o valor do erro após isso, em contrapartida, o método \textit{Linspace} continua a melhorar conforme o número de pontos aumenta, chegando ser o melhor a partir do 1000º ponto. 

\begin{figure}[H]
	\centering
	\begin{subfigure}[b]{0.45\textwidth}
		\centering 
		\includegraphics[width=\textwidth]{./figuras/ERRORPLOT_L1_TRUE_NORMAL_NEAREST_050_log}
		\caption{}
		\label{fig:error_norm_near_50}
	\end{subfigure}
	\hfill
	~ %add desired spacing between images, e. g. ~, \quad, \qquad, \hfill etc. 
	%(or a blank line to force the subfigure onto a new line)
	\begin{subfigure}[b]{0.45\textwidth}
		\centering 
		\includegraphics[width=\textwidth]{./figuras/ERRORPLOT_L1_TRUE_NORMAL_LINEAR_050_log}
		\caption{}
		\label{fig:error_norm_lin_50}
	\end{subfigure}
\caption{Erro total de estimação para a distribuição Normal com outlier em 50 (a) para a interpolação pelo vizinho mais próximo, (b) para a interpolação linear.}
\label{fig:Error_out}
\end{figure}
%\chapter{Conclusão} \label{cap:conclusao}
\vspace{-2cm}
A discretização contínua de dados representa uma importante tarefa de pré-processamento no contexto de estimação de dados. Até agora, muito trabalho foi feito para melhorar a discretização, a fim de reduzir as informações redundantes ou desconectadas.

Este trabalho abordou o problema de discretização no cenário de estimação de densidades, fazendo uma avaliação cuidadosa do assunto na literatura, propondo quatro diferentes métodos de discretização e comparando-os com o método \textit{Linspace}, amplamente utilizado na literatura. Em particular, o início e o final da região da variável aleatória devem ser definidos antes de aplicar a discretização, cujo desempenho é altamente sensível a esses parâmetros. Nesse contexto, pode ser importante procurar métodos mais resilientes.

Vale resaltar que estes métodos se comportaram de maneiras diferentes quando submetidos a funções analíticas ou com dados gerados, sendo que este último se agrava em distribuições assimétricas ou com caldas muito longas devido ao ruído que é inserido quando se deriva uma função discreta que já é ruidosa, e, curiosamente, este ruído contribuiu para que estes métodos (\ac{iPDF1} e \ac{iPDF2}) apresentassem resultados melhores que os esperados, fazendo-os uma boa opção de discretização apesar de seu algorítimo ser um pouco mais complexo de se implementar se comparado aos métodos \textit{Linspace} e \ac{CDFm}.

\section{Próximos Passos}
Ao final desse estudo, fica claro que existem possibilidades de melhorias nos métodos apresentados, tendo a necessidade de uma análise mais profunda para outras distribuições além das Normais e Lognormais além de se buscar outros métodos complementares aos já abordados ou até mesmo algum método que junte dois ou mais métodos, fazendo com que se pegue a melhor região para cada um deles.

%%%%%%%%%%%%%%%%%%%%%%%%%%%%%%%%%
%                               %
%         P\'{o}s textuais          %
%                               %
%%%%%%%%%%%%%%%%%%%%%%%%%%%%%%%%%

\bibliographystyle{abnt-alf}	     % Existem ainda: abbrv,acm,alpha,amsalpha e amsplain
\bibliography{./referencias/bibliografia}  % o nome do arquivo .bib com as refer\^{e}ncias

\appendix

%\include{./postextuais/apendice/apendiceA}
%\include{./postextuais/apendice/apendiceB}
%
%\include{/postextuais/apendice/apendiceC}

\end{document}

%%%%%%%%%%%%%%%%%%%%%%%%%%%%%%%%%%%%%%%%%%%%%%%%%%%%%%%%%%%%%%%%%%%%%%%%%%%%%%%%%%%%%%%%%%%%%%%%%%%%%%%%%%
%
% Resumo
%
%%%%%%%%%%%%%%%%%%%%%%%%%%%%%%%%%%%%%%%%%%%%%%%%%%%%%%%%%%%%%%%%%%%%%%%%%%%%%%%%%%%%%%%%%%%%%%%%%%%%%%%%%%
\chapter*{Resumo}
\vspace{-2cm}

\noindent 
\vspace{0.5cm}

Ultimamente, com o surgimento de grandes experimentos geradores de dados, há uma demanda crescente para otimizar os algoritmos responsáveis por interpretar esse volume de informações, de modo que ele use o mínimo de dados possível para realizar a operação desejada. Este trabalho permeia esse contexto, propondo alternativas em uma das escolhas mais elementares em algoritmos de estimação/classificação: a discretização de uma determinada variável. Esta pesquisa propõe avaliar as características de diferentes métodos de discretização aplicados à estimação da função de densidade de probabilidade considerando o trade-off entre desempenho e simplicidade, bem como a suscetibilidade a outliers. Além disso, este trabalho analisa as vantagens e desvantagens de cada método e indica possíveis formas de ampliar o conhecimento sobre o assunto abordado.

\noindent Palavras-chave:  Estimação de Densidade, Discretização, Estimação Não Paramétrica.\\

\newpage



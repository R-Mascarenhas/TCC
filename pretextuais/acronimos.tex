%\begin{siglas} %%ALTERAR OS EXEMPLOS ABAIXO, CONFORME A NECESSIDADE
\chapter*{LISTA DE ABREVIATURAS E SIGLAS}
\vspace{-1.5cm}
	\begin{acronym}
		\acro{AMISE}{Erro Quadrático Integrado Assintótico Médio (do inglês, \emph{Asymptotic Mean Integrated Squared Error}}
		\acro{AMSE}{Erro Quadrático Médio Assintótico (do inglês, \emph{Asymptotic Mean Squared Error}}
		\acro{ATLAS}{Dispositivo Instrumental Toroidal para o LHC (do inglês, \textit{A Toroidal LHC Apparatus})}
		\acro{BCV}{Validação cruzada tendenciosa (do inglês, \emph{Biased Cross-Validation}}
		\acro{CDF}{Função de Distribuição Cumulativa (do inglês, \textit{Cumulative Distribution Function})}
		\acro{CDFm}{Método CDF (do inglês, \textit{CDF method})}
		\acro{CERN}{Centro Europeu de Pesquisa Nuclear, (do francês, \textit{Conseil Européen pour la Recherche Nucléaire})}
		\acro{CMS}{Solenoide de Múon Compacto (do inglês, \emph{Compact Muon Solenoid})}
		\acro{FD}{Estimador Freedman–Diaconi}
		\acro{IAE}{Erro Absoluto Integrado (do inglês,\emph{Integrated Absolute Error}}
		\acro{iPDF1}{Integral da distribuição da primeira derivada da PDF}
		\acro{iPDF2}{Integral da distribuição da segunda derivada da PDF}
		\acro{ISE}{Erro Quadrado Integrado (do inglês,\emph{Integrated Squared Error}}
		\acro{KDE}{Estimação de Densidade de Núcleo, (do inglês, \textit{Kernel Density Estimation})}
		\acro{LHC}{Grande Colisor de Hádrons (do inglês, \textit{Large Hadron Collider})}
		\acro{LSCV}{Validação Cruzada pelo Mínimo Quadrado (do inglês, \emph{Least-Square Cross-Validation}}
		\acro{MISE}{Erro Quadrático Integrado Médio (do inglês, \emph{Mean Integrated Squared Error}}
		\acro{MKDE}{Estimativa Multivariada de Densidade de Kernel (do inglês, \emph{Multivariate Kernel Density Estimation)}}
		\acro{MSE}{Erro Quadrático Médio (do inglês, \emph{Mean Squared Error}}
		\acro{MVA}{Análise Multivariada, (do inglês, \emph{Multivariate Analysis})}
		\acro{PDF}{Função de Densidade de Probabilidade (do inglês, \textit{Probability Density Function})}
		\acro{PDFm}{Método PDF (do inglês, \textit{PDF method})}	
		\acro{RoI}{Regiões de Interesse (do inglês, \textit{Regions of Interest})}
	\end{acronym}

%\end{siglas}
%%%%%%%%%%%%%%%%%%%%%%%%%%%%%%%%%%%%%%%%%%%%%%%%%%%%%%%%%%%%%%%%%%%%%%%%%%%%%%%%%%%%%%%%%%%%%%%%%%%%%%%%%%
%
% Resumo
%
%%%%%%%%%%%%%%%%%%%%%%%%%%%%%%%%%%%%%%%%%%%%%%%%%%%%%%%%%%%%%%%%%%%%%%%%%%%%%%%%%%%%%%%%%%%%%%%%%%%%%%%%%%
\chapter*{Abstract}


\noindent 

Lately, with the emergence of large data-generating experiments, there is a growing demand to optimize the algorithms responsible for interpreting this volume of information so that it uses as little data as possible to perform the desired operation. This work permeates this context, proposing alternatives in one of the most elementary choices in estimation/classification algorithms: discretization of a given variable. This paper proposes to evaluate the characteristics of different discretization methods applied to probability density function estimation considering the trade-off between performance and simplicity, as well as susceptibility to outliers. In addition, this work analyzes the advantages and disadvantages of each method and indicates possible ways to extend the knowledge about the addressed subject.
\vspace{0.5cm}

\noindent Keywords: Density Estimation, Discretization, Non-parametric Estimation. \\

\newpage

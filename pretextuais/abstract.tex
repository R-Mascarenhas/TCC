%%%%%%%%%%%%%%%%%%%%%%%%%%%%%%%%%%%%%%%%%%%%%%%%%%%%%%%%%%%%%%%%%%%%%%%%%%%%%%%%%%%%%%%%%%%%%%%%%%%%%%%%%%
%
% Resumo
%
%%%%%%%%%%%%%%%%%%%%%%%%%%%%%%%%%%%%%%%%%%%%%%%%%%%%%%%%%%%%%%%%%%%%%%%%%%%%%%%%%%%%%%%%%%%%%%%%%%%%%%%%%%
\chapter*{Abstract}


\noindent The particle identification has fundamental importance to the high-energy physics experiments developed around the world. In this environment of particle physics, the particles occurrence probability relevant to the proposed studies is very low in relation to the particles that form the background noise, requiring algorithms with efficiency indices of interest signals detection and background noise rejection each time best. Many methods of identifying particles using the likelihood technique, which uses the probability distribution of variables to create a discriminator. Therefore, to guarantee the performance of the classification, a good quality of estimation of the distributions in question is necessary, being these commonly different from the distributions known and parameterized in the literature. In this thesis, the methods applied to the estimation of non-parametric density will be reviewed and possible optimizations will be evaluated from the data produced by one of the largest experiments of CERN, the ATLAS. Concentrated in the offline context, the paper reproduces the likelihood-based method and proposes some improvements with the use of more optimized processing algorithms, robust statistics and discretization techniques different from those used in the literature. In addition, this work proposes to expand the tools used in the univariate estimation for the multivariate density estimation environment, known as MKDE (Multivariate Kernel Density Estimation), which may be able to mitigate the inserted error in the consideration of independence between the discriminant variables inserted by the method of Likelihood currently in use by several experiments of this type. Initially, this work proposes to implement the likelihood method based on the estimation of univariate densities used in the reconstruction of the joint density of the discriminant variables and to study the impact of possible parameters related to the implementation of the algorithm for estimating univariate densities. In a second step, the implementation of MKDE is inserted through a direct comparison with the univariate method.


\vspace{0.5cm}

\noindent Keywords: Density estimation, Likelihood, FastKDE, Discretization. \\

\newpage

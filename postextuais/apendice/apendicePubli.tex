%%%%%%%%%%%%%%%%%%%%%%%%%%%%%%%%%%%%%%%%%%%%%%%%%%%%%%%%%%%%%%%%%%%%%%%%%%%%%%%%%%%%%%%%%%%%%%%%%%%%%%%%%%
%
% Ap\^{e}ndice - Anexo C
%
%%%%%%%%%%%%%%%%%%%%%%%%%%%%%%%%%%%%%%%%%%%%%%%%%%%%%%%%%%%%%%%%%%%%%%%%%%%%%%%%%%%%%%%%%%%%%%%%%%%%%%%%%%

\chapter{Lista de Publicações}\label{sec:apendicePubli}
\vspace{-2cm}

\section{Publicações em Anais de Congresso Internacional}

\begin{enumerate}

\item  COSTA, R. M., SOUZA, D. M., COSTA, I. A., NÓBREGA, R. A. "Study of the Discretization Process applied to Continuous Random Variables in the Density Estimation Context." Instrumentation Systems, Circuits and Transducers (INSCIT), 2018 3rd International Symposium on IEEE, 2018.

Ultimamente, com o surgimento de grandes experimentos geradores de dados, há uma demanda crescente para otimizar os algoritmos responsáveis por interpretar esse volume de informações, de modo que ele use o mínimo de dados possível para realizar a operação desejada. Este trabalho permeia esse contexto, propondo alternativas em uma das escolhas mais elementares em algoritmos de estimação/classificação: a discretização de uma determinada variável. Este artigo propõe avaliar as características de diferentes métodos de discretização aplicados à estimação da função de densidade de probabilidade considerando o trade-off entre desempenho e simplicidade, bem como a suscetibilidade a \textit{outliers}. Além disso, este trabalho analisa as vantagens e desvantagens de cada método e indica possíveis formas de ampliar o conhecimento sobre o assunto abordado.

\end{enumerate}

